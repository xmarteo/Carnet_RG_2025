% !TEX TS-program = lualatex
% !TEX encoding = UTF-8

\documentclass[Carnet_RG_2025.tex]{subfiles}

\ifcsname preamble@file\endcsname
  \setcounter{page}{\getpagerefnumber{M-20251129_02_Laudes}}
\fi

\begin{document}
\bigtitle{Laudes de la Sainte Vierge le samedi}{Samedi 29 novembre}{Messe}

\smallscore{or_dia_ferialis}{}

\gscore{an_filii_sion}{Les enfants de Sion exultent de joie en leur Roi.}
\psalmVulgate{149}

\gscore{an_quam_magnificata}{Qu'elles sont grandes, tes œuvres, Seigneur!}
\psalmVulgate{91}

\gscore{an_laetabitur_iustus}{Le juste se réjouira dans le Seigneur, et espérera en lui.}
\psalmVulgate{63}

\gscore{an_ostende_nobis}{Montre-nous, Seigneur, la lumière de ta miséricorde}
\canticumVulgate{Ecclesiastici}{de l'Ecclésiastique}

\gscore{an_omnis_spiritus}{Que tout ce qui respire loue le Seigneur.}
\psalmVulgate{150}

\capitulum{Eccl. 24: 14}{Ab inítio et ante sǽcula creá\sylprep{ta} \sylprep{sum},~\pscross{} et usque ad futúrum sǽcu\sylprep{lum} \sylprep{non} \sylac{dé}sinam,~\psstar{} et in habitatióne sancta coram ipso ministrávi.}{Dès le commencement, avant les siècles, il m’a créée, et pour les siècles je subsisterai ;
dans la demeure sainte, j’ai assuré mon service en sa présence.}

\smallscore{or_capitulum}{}

\gscore{hy_o_gloriosa_domina}{}
\translation{\colored{O} glorieuse femme,
élevée au-dessus des astres,
vous nourrissez du lait de votre sein
Celui qui vous a créée, devenu petit enfant.\\
\colored{V}ous nous rendez par votre auguste Fils,
ce dont Ève nous avait malheureusement privés :
vous ouvrez les portes du ciel
pour y faire entrer ceux qui pleurent.\\
\colored{V}ous êtes la porte du grand Roi,
et son palais, éclatant de lumière :
Nations rachetées, célébrez toutes la vie
qui nous est donnée par cette Vierge.\\
\colored{G}loire soit à vous, ô Jésus !
Qui êtes né de la Vierge :
Gloire au Père et à l’Esprit-Saint,
Dans les siècles éternels.
Ainsi soit-il.}

\versiculus{Benedícta tu in muliéribus.}{Et benedíctus fructus ventris tui.}{Vous êtes bénie entre les femmes.}{Et le fruit de Votre sein est béni.}

\gscore{an_beata_dei_genitrix}{Heureuse Mère de Dieu, Marie toujours Vierge, temple du Seigneur, sanctuaire du Saint-Esprit, seule plus que toute autre Vous avez plu à Jésus-Christ notre Seigneur. Priez pour le peuple, intervenez pour le clergé, intercédez pour les femmes consacrées à Dieu.}
\canticumVulgate{Benedictus}{de Zacharie}

\dominusvobiscum

\oratio{
Concéde nos fámulos tuos, quǽsumus, Dómine Deus, perpétua mentis et córporis sanitáte gaudére:~\pscross{} et, gloriósa beátæ Maríæ semper Vírginis intercessióne,~\psstar{} a præsénti liberári tristítia, et ætérna pérfrui lætítia.\\
Per Dóminum nostrum Iesum Christum, Fílium tuum: qui tecum vivit et regnat in unitáte Spíritus Sancti, Deus, per ómnia sǽcula sæculórum.
}{
Seigneur notre Dieu, accordez, s’il vous plaît, à nous vos serviteurs, de jouir d’une perpétuelle santé de l’âme et du corps : et grâce à la glorieuse intercession de la bienheureuse Marie toujours Vierge, d’être délivrés des tristesses du temps présent, puis de goûter les joies éternelles.\\
Par Notre Seigneur Jésus Christ, Votre Fils, qui vit et règne avec Vous et le Saint-Esprit, Dieu, maintenant et pour les siècles des siècles.
}

\rubric{Commémoraison de S. Saturnin, martyr}

\gscore{an_qui_odit}{Celui qui méprise son âme en ce monde, la conserve pour la vie éternelle.}

\versiculus{Iustus ut palma florébit.}{Sicut cedrus Líbani multiplicábitur.}{Le juste fleurira comme le palmier.}{Il croîtra comme le cèdre du Liban.}

\oratio{
Deus, qui nos beáti Saturníni Mártyris tui concédis natalício pérfrui:~\pscross{} eius nos tríbue méritis adiuvári.\\
Per Dóminum nostrum Iesum Christum, Fílium tuum:~\psstar{} qui tecum vivit et regnat in unitáte Spíritus Sancti, Deus,~\pscross{} per ómnia sǽcula sæculórum.
}{
Dieu, qui nous faites la grâce de nous réjouir de la naissance céleste du bienheureux Saturnin, votre Martyr, accordez-nous d’être secourus par ses mérites.\\
Par Jésus-Christ, ton Fils, notre Seigneur, qui vit et règne avec toi dans l’unité du Saint-Esprit, Dieu, pour les siècles des siècles.
}

\dominusvobiscum

\smallscore{or_benedicamus_FE_BVM_sabbato}{\vvrub Bénissons le Seigneur. \rrrub Nous rendons grâces à Dieu.}

\fideliumanimae

\end{document}