% !TEX TS-program = lualatex
% !TEX encoding = UTF-8

\documentclass[Carnet_RG_2025.tex]{subfiles}

\ifcsname preamble@file\endcsname
  \setcounter{page}{\getpagerefnumber{M-20251130_02_Laudes}}
\fi

\begin{document}
\bigtitle{Laudes du premier dimanche de l'Avent}{Dimanche 30 novembre}{Laudes}


\gscore{or_dia_FO_festivus}{} % ou solennel ?

\gscore{hy_vox_clara_ecce_intonat}{}
\hymntranslation{TODO}

\gscore{an_in_illa_die}{En ce jour-là, les montagnes distilleront la douceur, et les collines feront couler le lait et le miel, alléluia.}
\psalmNova{62}

\gscore{an_montes_et_colles}{TODO}
\canticumNova{TriumPuerorum}{des trois enfants}

\gscore{an_ecce_veniet}{INSERT TRANSLATION HERE} % TODO
\psalmNova{149}

\lectiobrevis{1 Th. 5: 23-24}{TODO LA}{TODO FR}

\gscore{rb_christe_fili_dei}{\rr INSERT TRANSLATION HERE \\\vv INSERT TRANSLATION HERE} %% TODO attention: ce RBr a plusieurs tons, dont un spécifique pour l'Avent, sauf erreur de ma part

\gscore{an_dies_domini}
\canticumNova{Benedictus}{de Zacharie}

\smallscore{ORPreces} % A mettre à jour
% TODO PRECES:
\twocoltext{TODO LA}{TODO FR}

\smallscore{or_pater_festivus}{} % à adapter si nécessaire

\oratio{TODO LA}{TODO FR}

\blessing

\smallscore{or_benedicamus_adventus}{\vvrub Bénissons le Seigneur. \rrrub Nous rendons grâces à Dieu.}

\end{document}