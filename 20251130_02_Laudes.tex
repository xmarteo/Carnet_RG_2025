% !TEX TS-program = lualatex
% !TEX encoding = UTF-8

\documentclass[Carnet_RG_2025.tex]{subfiles}

\ifcsname preamble@file\endcsname
  \setcounter{page}{\getpagerefnumber{M-20251130_02_Laudes}}
\fi

\begin{document}
\bigtitle{Laudes du premier dimanche de l'Avent}{Dimanche 30 novembre}{Laudes}


\gscore{or_dia_festivus}{} % ou solennel ?

\gscore{hy_vox_clara_ecce_intonat}{}
\translation{TODO}

\gscore{an_in_illa_die}{INSERT_TRANSLATION_HERE} % TODO
\psalmNova{62}

\gscore{an_montes_et_colles}{INSERT_TRANSLATION_HERE} % TODO
\canticumNova{Dn3}{des trois enfants}

\gscore{an_ecce_veniet}{INSERT_TRANSLATION_HERE} % TODO
\psalmNova{149}

\lectiobrevis{1 Th. 5: 23-24}{TODO_LA}{TODO_FR}

\gscore{rb_christe_fili_dei}{\rr INSERT_TRANSLATION_HERE \\\vv INSERT_TRANSLATION_HERE} %% TODO attention: ce RBr a plusieurs tons, dont un spécifique pour l'Avent, sauf erreur de ma part

\gscore{an_dies_domini}
\canticumNova{Ben}{de Zacharie}

\smallscore{ORPreces} % A mettre à jour
% TODO PRECES:
\twocoltext{TODO_LA}{TODO_FR}

\smallscore{or_pater_festivus}{} % à adapter si nécessaire

\oratio{TODO_LA}{TODO_FR}

\blessing

\gscore{or_benedicamus_adventus} % TODO

\end{document}