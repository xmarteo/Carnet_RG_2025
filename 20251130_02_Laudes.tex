% !TEX TS-program = lualatex
% !TEX encoding = UTF-8

\documentclass[Carnet_RG_2025.tex]{subfiles}

\ifcsname preamble@file\endcsname
  \setcounter{page}{\getpagerefnumber{M-20251130_02_Laudes}}
\fi

\begin{document}
\bigtitle{Laudes du premier dimanche de l'Avent}{Dimanche 30 novembre}{Laudes}


\gscore{or_dia_FO_festivus}{} % ou solennel ?

\gscore{hy_vox_clara_ecce_intonat}{}
\hymntranslation{TODO}

\gscore{an_in_illa_die}{\aarub En ce jour-là, les montagnes distilleront la douceur, et les collines feront couler le lait et le miel, alléluia.}
\psalmNova{62}

\gscore{an_montes_et_colles}{\aarub TODO}
\canticumNova{TriumPuerorum}{des trois enfants}

\gscore{an_ecce_veniet}{\aarub INSERT TRANSLATION HERE} % TODO
\psalmNova{149}

\lectiobrevis{Rm. 13, 11-12}{Hora est iam vos de somno súrgere, nunc enim
própior est nobis salus quam cum credídimus. Nox
procéssit, dies autem appropiávit. Abiciámus ergo
ópera tenebrárum et induámur arma lucis.}{C’est le moment, l’heure est venue de sortir de votre sommeil. Car le salut est plus près de nous maintenant qu’à l’époque où nous sommes devenus croyants. La nuit est bientôt finie, le jour est tout proche. Rejetons les activités des ténèbres, revêtons-nous pour le combat de la lumière. }

\gscore{rb_christe_fili_dei}{\rrrub Christ, Fils du Dieu vivant, aie pitié de nous. \vvrub Toi qui dois venir dans le monde.} %% TODO vérifier que c'est le bon ton

\gscore{an_dies_domini}{\aarub TODO traduction}
\canticumNova{Benedictus}{de Zacharie}

\smallscore{ORPreces} % A mettre à jour
% TODO PRECES:
\twocoltext{TODO LA}{TODO FR}

\smallscore{or_pater_festivus}{} % à adapter si nécessaire

\oratio{TODO LA}{TODO FR}

\blessing

\smallscore{or_benedicamus_adventus}{\vvrub Bénissons le Seigneur. \rrrub Nous rendons grâces à Dieu.}

\end{document}