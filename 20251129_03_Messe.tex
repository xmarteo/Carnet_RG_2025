% !TEX TS-program = lualatex
% !TEX encoding = UTF-8

\documentclass[Carnet_RG_2025.tex]{subfiles}

\ifcsname preamble@file\endcsname
  \setcounter{page}{\getpagerefnumber{M-20251129_03_Messe}}
\fi

\begin{document}
\bigtitle{Messe votive de la Sainte Vierge}{Samedi 29 novembre}{Messe}

\smalltitle{Introit}

\gscore{in_salve_sancta_parens}{\aarub Nous te saluons, ô Mère sainte, qui as mis au monde le Roi qui règne sur le ciel et la terre, dans les siècles des siècles.
\vvrub De belles paroles jaillissent de mon cœur, je dis mes œuvres au Roi.}

\gscore{ky_K10}{}

\gscore{ky_G10}{}
\hymntranslation{\emph{Gloire à Dieu, au plus haut des cieux,
et paix sur la terre aux hommes qu’il aime.
Nous te louons, nous te bénissons, nous t’adorons,
Nous te glorifions, nous te rendons grâce, pour ton immense gloire,
Seigneur Dieu, Roi du ciel, Dieu le Père tout-puissant.
Seigneur, Fils unique, Jésus Christ,
Seigneur Dieu, Agneau de Dieu, le Fils du Père.
Toi qui enlèves le péché du monde, prends pitié de nous
Toi qui enlèves le péché du monde, reçois notre prière ;
Toi qui es assis à la droite du Père, prends pitié de nous.
Car toi seul es saint, toi seul es Seigneur,
Toi seul es le Très-Haut, Jésus Christ, avec le Saint-Esprit
Dans la gloire de Dieu le Père. Amen.}}

\smalltitle{Collecte}

\twocoltext{
\vv Dóminus vobíscum.\\
\rr Et cum spíritu tuo.\\
\vv Orémus.\\
Concéde nos fámulos tuos, quǽsumus, Dómine Deus, perpétua mentis et córporis sanitáte gaudére: et, gloriósa beátæ Maríæ semper Vírginis intercessióne, a præsénti liberári tristítia, et ætérna pérfrui lætítia.
Per Dóminum nostrum Iesum Christum, Fílium tuum: qui tecum vivit et regnat in unitáte Spíritus Sancti Deus, per ómnia sǽcula sæculórum.\\
\rr Amen.
}{
\vv Le Seigneur soit avec vous. \\
\rr Et avec votre esprit.\\
\vv Prions.\\
Accorde-nous, Seigneur, une heureuse santé de l'âme et du corps, et par la glorieuse intercession de la bienheureuse Marie toujours vierge, délivre-nous des tristesses présentes, et conduis-nous à la joie éternelle.
Par Jésus-Christ, ton Fils, notre Seigneur, qui vit et règne avec toi dans l’unité du Saint-Esprit, Dieu, pour les siècles des siècles.\\
\rr Amen.
}

\pagebreak

\rubric{Commémoration de saint Saturnin, martyr}

\twocoltext{
\vv Orémus.\\
Deus, qui nos beáti Saturníni Mártyris tui concédis natalícia pérfrui: eius nos tríbue méritis adiuvári.
Per Dóminum nostrum Iesum Christum, Fílium tuum: qui tecum vivit et regnat in unitáte Spíritus Sancti Deus, per ómnia sǽcula sæculórum.\\
\rr Amen.
}{
\vv Prions.\\
Dieu, qui nous fais la grâce de nous réjouir en la fête du bienheureux Saturnin, ton martyr : accorde-nous d’être secourus par ses mérites.
Par notre Seigneur Jésus-Christ, ton Fils, qui, étant Dieu, vit et règne avec toi, dans l’unité du Saint-Esprit, pour les siècles des siècles.\\
\rr Amen.
}

\smalltitle{Épître}

\twocoltext{
Léctio libri Sapiéntiæ.\\
\rubric{Si. 24 : 9-12}\\
Ab inítio et ante sǽcula creáta sum, et usque ad futúrum sǽculum non désinam, et in habitatióne sancta coram ipso ministrávi. Et sic in Sion firmáta sum, et in civitáte sanctificáta simíliter requiévi, et in Ierúsalem potéstas mea. Et radicávi in pópulo honorificáto, et in parte Dei mei heréditas illíus, et in plenitúdine sanctórum deténtio mea.

\rubric{À la messe chantée, on ne répond rien.}
}{
Lecture du livre de la Sagesse.\\
Dès le commencement, avant les siècles, il m’a créée, et pour les siècles je subsisterai ;
dans la demeure sainte, j’ai assuré mon service en sa présence. Ainsi, je me suis fixée dans Sion,
il m’a fait demeurer dans la cité bien-aimée, et dans Jérusalem j’exerce ma puissance.
Je me suis enracinée dans un peuple glorieux, dans le domaine du Seigneur, dans son héritage : j’habite au milieu de l’assemblée des saints.
}

\gscore{gr_benedicta_et_venerabilis}{\rrrub Tu es bénie et vénérable, Vierge Marie, toi qui, sans avoir rien perdu de ton intégrité corporelle, es devenue la mère du Sauveur. \vvrub Vierge, Mère de Dieu, celui que l'univers entier ne peut contenir, devenu homme, s'est enfermé dans ton sein.}

\gscore{al_post_partum}{\vvrub Après l'enfantement, ô Vierge, tu es demeurée inviolée: Mère de Dieu, intercède pour nous, alléluia.}

\pagebreak

\smalltitle{Évangile}

\twocoltext{
\vv Dóminus vobíscum.\\
\rr Et cum spíritu tuo.\\
\vv Sequéntia \cc sancti Evangélii secúndum Lucam.\\
\rr Glória tibi, Dómine.
}{
\vv Le Seigneur soit avec vous.\\
\rr Et avec votre esprit.\\
\vv Suite du saint Évangile selon saint Luc.\\
\rr Gloire à toi, Seigneur.
}
\twocoltext{
\rubric{Lc 11 : 27-28}\\
In illo témpore: Loquénte Iesu ad turbas, extóllens vocem quædam múlier de turba, dixit illi: Beátus venter, qui te portávit, et úbera, quæ suxísti. At ille dixit: Quinímmo beáti, qui áudiunt verbum Dei, et custódiunt illud.

\rubric{À la messe chantée, on ne répond rien.}
}{
En ce temps-là, Jésus parlait aux foules, quand une femme éleva la voix du milieu de la foule et lui dit : « Heureux le ventre qui t'a porté, et les seins qui t'ont allaité ! » Mais lui répondit : « Heureux plutôt ceux qui écoutent la Parole de Dieu et qui la gardent. »
}

\smalltitle{Offertoire}

\twocoltext{
\vv Dóminus vobíscum.\\
\rr Et cum spíritu tuo.\\
\vv Orémus.
}{
\vv Le Seigneur soit avec vous.\\
\rr Et avec votre esprit.\\
\vv Prions.
}

\gscore{of_ave_maria}{\rubric{Lc 1: 28, 42} \rrrub Je te salue, Marie, pleine de grâce, le Seigneur est avec toi, tu êtes bénie entre toutes les femmes, et le fruit de tes entrailles est béni. \vvrub Comment cela se fera-il, puisque je ne connais pas d'homme? L'Esprit du Seigneur viendra sur toi, et la puissance du Très-Haut te couvrira. \vvrub C'est pourquoi celui qui naîtra sera saint, il sera appelé Fils de Dieu.}

\twocoltext{
Súscipe, sancte Pater, omnípotens ætérne Deus, hanc immaculátam hóstiam, quam ego indígnus fámulus tuus óffero tibi Deo meo vivo et vero, pro innumerabílibus peccátis, et offensiónibus, et neglegéntiis meis, et pro ómnibus circumstántibus, sed et pro ómnibus fidélibus christiánis vivis atque defúnctis: ut mihi, et illis profíciat ad salútem in vitam ætérnam. Amen.
}{
Recevez, Père saint, Dieu éternel et tout-puissant, cette hostie sans tache, que moi, votre indigne serviteur, je vous offre à vous, mon Dieu vivant et vrai, pour mes innombrables péchés, offenses et négligences, pour tous ceux qui m’entourent, ainsi que pour tous les fidèles chrétiens vivants et morts, afin qu’elle serve à mon salut et au leur pour la vie éternelle. Ainsi soit-il.
}

\twocoltext{
Deus, qui humánæ substántiæ dignitátem mirabíliter condidísti, et mirabílius reformásti: da nobis per huius aquæ et vini mystérium, eius divinitátis esse consórtes, qui humanitátis nostræ fíeri dignátus est párticeps, Iesus Christus, Fílius tuus, Dóminus noster: Qui tecum vivit et regnat in unitáte Spíritus Sancti Deus: per ómnia sǽcula sæculórum. Amen.
}{
Dieu, qui avez admirablement fondé la dignité de la nature humaine et l’avez restaurée plus admirablement encore : donnez-nous, par le mystère de cette eau et de ce vin, d’avoir part à la divinité de celui qui a daigné partager notre humanité, Jésus-Christ, votre Fils, notre Seigneur, qui, étant Dieu, vit et règne avec vous dans l’unité du Saint-Esprit, dans tous les siècles des siècles. Ainsi soit-il.
}

\twocoltext{
Offérimus tibi, Dómine, cálicem salutáris, tuam deprecántes cleméntiam: ut in conspéctu divínæ maiestátis tuæ, pro nostra et totíus mundi salúte, cum odóre suavitátis ascéndat. Amen.
In spíritu humilitátis et in ánimo contríto suscipiámur a te, Dómine: et sic fiat sacrifícium nostrum in conspéctu tuo hódie, ut pláceat tibi, Dómine Deus.
Veni, sanctificátor omnípotens ætérne Deus: et béne dic hoc sacrifícium, tuo sancto nómini præparátum.
}{
Nous vous offrons, Seigneur, le calice du salut, implorant votre clémence : qu’il s’élève en odeur de suavité devant votre divine majesté, pour notre salut et celui du monde entier. Ainsi soit-il.
En esprit d’humilité et le cœur contrit, puissions-nous être accueillis par vous, Seigneur : et que notre sacrifice ait lieu aujourd’hui devant vous de telle manière qu’il vous soit agréable, Seigneur Dieu.
Venez, Sanctificateur, Dieu éternel et tout-puissant, et bénissez ce sacrifice préparé pour la gloire de votre saint Nom.
}

\twocoltext{
Lavábo inter innocéntes manus meas: et circúmdabo altáre tuum, Dómine: Ut áudiam vocem laudis, et enárrem univérsa mirabília tua. Dómine, diléxi decórem domus tuæ et locum habitatiónis glóriæ tuæ. Ne perdas cum ímpiis, Deus, ánimam meam, et cum viris sánguinum vitam meam: In quorum mánibus iniquitátes sunt: déxtera eórum repléta est munéribus. Ego autem in innocéntia mea ingréssus sum: rédime me et miserére mei. Pes meus stetit in dirécto: in ecclésiis benedícam te, Dómine.
Glória Patri, et Fílio, et Spirítui Sancto. Sicut erat in princípio, et nunc, et semper, et in sǽcula sæculórum. Amen.
}{
Je laverai mes mains parmi les innocents, et je me tiendrai autour de Votre autel, Seigneur. Pour entendre la voix de Vos louanges, et pour raconter toutes Vos merveilles. Seigneur, j'ai aimé la beauté de Votre maison, et le lieu où habite Votre gloire. Ne perdez pas, ô Dieu, mon âme avec les impies, ni ma vie avec les hommes de sang qui ont l'iniquité dans les mains, et dont la droite est remplie de présents. Pour moi j'ai marché dans mon innocence : délivrez-moi et ayez pitié de moi. Mon pied s’est tenu dans le droit chemin : je Vous bénirai, Seigneur, dans les assemblées. Gloire au Père et au Fils, et au Saint-Esprit. Comme il était au commencement, maintenant et toujours, et dans les siècles des siècles. Ainsi soit-il.
}

\twocoltext{
Súscipe, sancta Trínitas, hanc oblatiónem, quam tibi offérimus ob memóriam passiónis, resurrectiónis, et ascensiónis Iesu Christi, Dómini nostri: et in honórem beátæ Maríæ semper Vírginis, et beáti Ioannis Baptistæ, et sanctórum Apostolórum Petri et Pauli, et istórum et ómnium Sanctórum: ut illis profíciat ad honórem, nobis autem ad salútem: et illi pro nobis intercédere dignéntur in cælis, quorum memóriam ágimus in terris. Per eúndem Christum, Dóminum nostrum. Amen.
}{
Recevez, Trinité Sainte, cette offrande que nous vous présentons en mémoire de la Passion, de la Résurrection et de l’Ascension de Jésus-Christ notre Seigneur ; et en l’honneur de la bienheureuse Marie toujours vierge, de saint Jean-Baptiste, des saints apôtres Pierre et Paul, de ceux-ci et de tous vos saints : qu’elle serve à leur honneur et à notre salut ; et qu’ils daignent intercéder au ciel pour nous qui faisons mémoire d’eux sur la terre. Par le même Christ notre Seigneur. Ainsi soit-il.
}

\smalltitle{Secrète}

\twocoltext{
\vv Oráte, fratres: ut meum ac vestrum sacrifícium acceptábile fiat apud Deum Patrem omnipoténtem.\\
\rr Suscípiat Dóminus sacrifícium de mánibus tuis ad laudem et glóriam nominis sui, ad utilitátem quoque nostram, totiúsque Ecclésiæ suæ sanctæ.\\
\vv Amen.

Tua, Dómine, propitiatióne, et beátæ Maríæ semper Vírginis intercessióne, ad perpétuam atque præséntem hæc oblátio nobis profíciat prosperitátem et pacem.\\
Per Dóminum.\\
\rr Amen.

\rubric{Commémoration de saint Saturnin, martyr}\\
Múnera, Dómine, tibi dicáta sanctífica: et, intercedénte beáto Saturníno Mártyre tuo, per hæc eádem nos placátus inténde.\\
Per Dóminum nostrum Iesum Christum, Fílium tuum: qui tecum vivit et regnat in unitáte Spíritus Sancti Deus, per ómnia sǽcula sæculórum.
}{
\vv Priez, mes frères, afin que mon sacrifice, qui est aussi le vôtre, soit agréé par Dieu le Père tout-puissant.\\
\rr Que le Seigneur reçoive de vos mains le sacrifice, à la louange et à la gloire de son nom, et aussi pour notre bien et celui de toute sa sainte Église.\\
\vv Ainsi soit-il.

Que par votre bonté, Seigneur, et par l'intercession de la bienheureuse Marie toujours vierge, l'offrande de ce sacrifice nous procure, pour l'éternité comme pour la vie présente, le bonheur et la paix.
Par notre Seigneur...\\
\rr Ainsi soit-il.

Sanctifiez, Seigneur, ces dons qui vous sont consacrés, grâce à eux et l’intercession du bienheureux Saturnin, votre Martyr, jetez sur nous un regard de paix et de bonté.
Par notre Seigneur Jésus-Christ, votre Fils, qui, étant Dieu, vit et règne avec vous, en l’unité du Saint-Esprit, dans tous les siècles des siècles.\\
\rr Ainsi soit-il.
}

\smalltitle{Préface}

\smallscore{MOR10_PrefaceOF}{\vvrub Le Seigneur soit avec vous. \rrrub Et avec votre esprit.\\
\vvrub Élevons nos cœurs. \rrrub Ils sont tournés vers le Seigneur.\\
\vvrub Rendons grâces au Seigneur notre Dieu. \rrrub Cela est digne et juste.}

\twocoltext{
Vere dignum et iustum est, æquum et salutáre, nos tibi semper et ubíque grátias ágere: Dómine sancte, Pater omnípotens, ætérne Deus: Et te in veneratione beátæ Maríæ semper Vírginis collaudáre, benedícere et prædicáre. Quæ et Unigénitum tuum Sancti Spíritus obumbratióne concépit: et, virginitátis glória permanénte, lumen ætérnum mundo effúdit, Iesum Christum, Dóminum nostrum. Per quem maiestátem tuam laudant Angeli, adórant Dominatiónes, tremunt Potestátes. Cæli cælorúmque Virtútes ac beáta Séraphim sócia exsultatióne concélebrant. Cum quibus et nostras voces ut admítti iubeas, deprecámur, súpplici confessióne dicéntes:
}{
Il est vraiment juste et nécessaire, c’est notre devoir et notre salut, de vous rendre grâces toujours et partout, Seigneur, Père saint, Dieu éternel et tout-puissant. Et de vous louer, bénir et célébrer en la fête de la bienheureuse Marie toujours vierge, elle qui, le Saint-Esprit la couvrant de son ombre, a conçu votre Fils unique et, sans perdre la gloire de la virginité, a mis au monde la lumière éternelle, Jésus-Christ notre Seigneur. C’est par lui que les Anges louent votre majesté, que les Dominations l’adorent, que les Puissances la révèrent, que les Cieux et les Vertus des cieux, ainsi que les bienheureux Séraphins, la célèbrent dans une même allégresse. À leurs chants nous vous supplions de laisser se joindre aussi nos voix, pour proclamer dans une humble louange:
}

\gscore{ky_S10}{Saint, Saint, Saint, le Seigneur, Dieu des Forces célestes. Le ciel et la terre sont remplis de votre Gloire. Hosanna au plus haut des cieux.
Béni soit celui Qui vient au Nom du Seigneur. Hosanna au plus haut des cieux.}

\smalltitle{Canon romain}

\twocoltext{
Te ígitur, clementíssime Pater, per Iesum Christum, Fílium tuum, Dóminum nostrum, súpplices rogámus, ac pétimus, uti accépta hábeas et benedícas, hæc \cc dona, hæc \cc múnera, hæc \cc sancta sacrifícia illibáta, in primis, quæ tibi offérimus pro Ecclésia tua sancta cathólica: quam pacificáre, custodíre, adunáre et régere dignéris toto orbe terrárum: una cum fámulo tuo Papa nostro et Antístite nostro et ómnibus orthodóxis, atque cathólicæ et apostólicæ fídei cultóribus.
}{
Père très clément, c’est donc vous que nous prions, suppliants, et à qui nous demandons, par Jésus-Christ votre Fils, notre Seigneur, d’accepter et de bénir ces dons, ces présents, ces offrandes saintes et immaculées.
Tout d’abord, nous vous les offrons pour votre sainte Église catholique : daignez lui donner la paix, la protéger, la réunir et la gouverner par toute la terre ; et en même temps pour votre serviteur notre Pape , notre évêque , tous ceux qui enseignent la vraie doctrine, et ceux qui gardent la foi catholique et apostolique.
}

\twocoltext{
Meménto, Dómine, famulórum famularúmque tuarum \rubric{N.} et \rubric{N.} et ómnium circumstántium, quorum tibi fides cógnita est et nota devótio, pro quibus tibi offérimus: vel qui tibi ófferunt hoc sacrifícium laudis, pro se suísque ómnibus: pro redemptióne animárum suárum, pro spe salútis et incolumitátis suæ: tibíque reddunt vota sua ætérno Deo, vivo et vero.
}{
Souvenez-vous, Seigneur, de vos serviteurs et de vos servantes \rubric{N.} et \rubric{N.}, et de tous ceux qui nous entourent : vous connaissez leur foi, vous avez éprouvé leur attachement. Nous vous offrons ou ils vous offrent eux-mêmes ce sacrifice de louange, pour eux et pour tous les leurs, pour la rédemption de leurs âmes, dans l’espérance de leur salut et de leur intégrité ; et ils vous adressent leurs prières, à vous, Dieu éternel, vivant et vrai.
}

\twocoltext{
Communicántes, et memóriam venerántes, in primis gloriósæ semper Vírginis Maríæ, Genetrícis Dei et Dómini nostri Iesu Christi: sed et beáti Ioseph, eiúsdem Vírginis Sponsi,
et beatórum Apostolórum ac Mártyrum tuórum, Petri et Pauli, Andréæ, Iacóbi, Ioánnis, Thomæ, Iacóbi, Philíppi, Bartholomǽi, Matthǽi, Simónis et Thaddǽi: Lini, Cleti, Cleméntis, Xysti, Cornélii, Cypriáni, Lauréntii, Chrysógoni, Ioánnis et Pauli, Cosmæ et Damiáni: et ómnium Sanctórum tuórum; quorum méritis precibúsque concédas, ut in ómnibus protectiónis tuæ muniámur auxílio. Per eúndem Christum, Dóminum nostrum. Amen.
}{
Unis dans une même communion, nous vénérons d’abord la mémoire de la glorieuse Marie toujours vierge, mère de notre Dieu et Seigneur Jésus-Christ, puis celle du bienheureux Joseph, l’époux de la Vierge,
de vos bienheureux apôtres et martyrs, Pierre et Paul, André, Jacques, Jean, Thomas, Jacques, Philippe, Barthélémy, Matthieu, Simon et Jude, Lin, Clet, Clément, Sixte, Corneille, Cyprien, Laurent, Chrysogone, Jean et Paul, Côme et Damien, et de tous vos saints.
À leurs prières et par leurs mérites, accordez-nous d’être fortifiés en toute occasion par le secours de votre protection. Par le même Christ notre Seigneur. Ainsi soit-il.
}

\twocoltext{
Hanc ígitur oblatiónem servitútis nostræ, sed et cunctæ famíliæ tuæ,
quǽsumus, Dómine, ut placátus accípias: diésque nostros in tua pace dispónas, atque ab ætérna damnatióne nos éripi, et in electórum tuórum iúbeas grege numerári. Per Christum, Dóminum nostrum. Amen.
}{
Cette oblation donc de notre ministère, mais aussi de votre famille entière,
nous vous prions, Seigneur, de l’accepter avec bienveillance, de disposer nos jours dans votre paix, et d’ordonner que nous soyons arrachés à la damnation éternelle et comptés dans la troupe de vos élus. Par le Christ notre Seigneur. Ainsi soit-il.
}

\twocoltext{
Quam oblatiónem tu, Deus, in ómnibus, quǽsumus, bene\cc{}díctam, adscríp\cc{}tam, ra\cc{}tam, rationábilem, acceptabilémque fácere dignéris: ut nobis Cor\cc{}pus, et San\cc{}guis fiat dilectíssimi Fílii tui, Dómini nostri Iesu Christi.
}{
Cette oblation, ô Dieu, nous vous en prions, daignez la rendre en tout point bénie, approuvée, ratifiée, digne et agréable : afin qu’elle devienne pour nous le Corps et le Sang de votre Fils bien-aimé, notre Seigneur Jésus-Christ.
}

\twocoltext{
Qui prídie quam paterétur, accépit panem in sanctas ac venerábiles manus suas, elevátis óculis in cælum ad te Deum, Patrem suum omnipoténtem, tibi grátias agens, bene\cc{}díxit, fregit, dedítque discípulis suis, dicens: Accípite, et manducáte ex hoc omnes.
}{
La veille du jour où il a souffert, il a pris du pain dans ses mains saintes et vénérables et, les yeux levés au ciel vers vous, Dieu son Père tout-puissant, vous rendant grâces, l’a béni, rompu et donné à ses disciples, en disant :
Prenez et mangez tous de ceci :
}

\twocoltext{
HOC EST ENIM CORPUS MEUM.
}{
CAR CECI EST MON CORPS.
}

\twocoltext{
Símili modo postquam cenátum est, accípiens et hunc præclárum Cálicem in sanctas ac venerábiles manus suas: item tibi grátias agens, bene\cc{}díxit, dedítque discípulis suis, dicens: Accípite, et bíbite ex eo omnes.
}{
De même, après le repas, prenant aussi ce très glorieux calice dans ses mains saintes et vénérables, vous rendant grâces encore, il l’a béni et donné à ses disciples, en disant :
« Prenez, et buvez-en tous :
}

\twocoltext{
HIC EST ENIM CALIX SANGUINIS MEI, NOVI ET ÆTERNI TESTAMENTI: MYSTERIUM FIDEI: QUI PRO VOBIS ET PRO MULTIS EFFUNDETUR IN REMISSIONEM PECCATORUM.
Hæc quotiescúmque fecéritis, in mei memóriam faciétis.
}{
CAR CECI EST LE CALICE DE MON SANG,
CELUI DE L’ALLIANCE NOUVELLE ET ÉTERNELLE
– MYSTÈRE DE LA FOI –
QUI SERA RÉPANDU POUR VOUS ET POUR BEAUCOUP
EN RÉMISSION DES PÉCHÉS.
Chaque fois que vous ferez cela, vous le ferez en mémoire de moi.
}

\twocoltext{
Unde et mémores, Dómine, nos servi tui, sed et plebs tua sancta, eiúsdem Christi Fílii tui, Dómini nostri, tam beátæ passiónis, nec non et ab ínferis resurrectiónis, sed et in cælos gloriósæ ascensiónis: offérimus præcláræ maiestáti tuæ de tuis donis ac datis, hóstiam \cc puram, hóstiam \cc sanctam, hóstiam \cc immaculátam, Panem \cc sanctum vitæ ætérnæ, et Cálicem \cc salútis perpétuæ.
}{
C’est pourquoi, Seigneur, nous vos serviteurs, et aussi votre peuple saint, en mémoire de la bienheureuse Passion de votre Fils Jésus-Christ notre Seigneur, de sa Résurrection des enfers et aussi de sa glorieuse Ascension dans les cieux, nous présentons à votre sublime majesté cette offrande venant des biens que vous nous avez donnés : la victime pure, la victime sainte, la victime immaculée, le Pain sacré de la vie éternelle et le Calice de l’éternel salut.
}

\twocoltext{
Supra quæ propítio ac seréno vultu respícere dignéris: et accépta habére, sicúti accépta habére dignátus es múnera púeri tui iusti Abel, et sacrifícium Patriárchæ nostri Abrahæ: et quod tibi óbtulit summus sacérdos tuus Melchísedech, sanctum sacrifícium, immaculátam hóstiam.
}{
Sur ces offrandes daignez jeter un regard favorable et serein, et les accepter comme vous avez bien voulu accepter les présents de votre serviteur Abel le Juste, le sacrifice de notre patriarche Abraham, et celui que vous offrit votre grand prêtre Melchisédech, sacrifice saint, victime immaculée.
}

\twocoltext{
Súpplices te rogámus, omnípotens Deus: iube hæc perférri per manus sancti Angeli tui in sublíme altáre tuum, in conspéctu divínæ maiestátis tuæ: ut, quotquot ex hac altáris participatióne sacrosánctum Fílii tui Cor\cc{}pus, et Sán\cc{}guinem sumpsérimus, omni benedictióne cælésti et grátia repleámur. Per eúndem Christum, Dóminum nostrum. Amen.
}{
Suppliants, nous vous en prions, Dieu tout-puissant : ordonnez que ces offrandes soient portées par les mains de votre saint Ange sur votre sublime autel, en présence de votre majesté divine ; afin que, nous tous qui recevrons par cette participation de l’autel le Corps et le Sang très saints de votre Fils, nous soyons comblés de toute grâce et bénédiction céleste. Par le même Christ notre Seigneur. Ainsi soit-il.
}

\twocoltext{
Meménto étiam, Dómine, famulórum famularúmque tuárum N. et N., qui nos præcessérunt cum signo fídei, et dórmiunt in somno pacis. Ipsis, Dómine, et ómnibus in Christo quiescéntibus locum refrigérii, lucis, et pacis, ut indúlgeas, deprecámur. Per eúndem Christum, Dóminum nostrum. Amen.
}{
Souvenez-vous aussi, Seigneur, de vos serviteurs et de vos servantes N. et N., qui nous ont précédés avec le signe de la foi, et qui dorment du sommeil de la paix. À eux, Seigneur, et à tous ceux qui reposent dans le Christ, nous vous supplions d’accorder le lieu du rafraîchissement, de la lumière et de la paix. Par le même Christ notre Seigneur. Ainsi soit-il.
}

\twocoltext{
Nobis quoque peccatóribus fámulis tuis, de multitúdine miseratiónum tuárum sperántibus, partem áliquam et societátem donáre dignéris, cum tuis sanctis Apóstolis et Martýribus: cum Ioánne, Stéphano, Matthía, Bárnaba, Ignátio, Alexándro, Marcellíno, Petro, Felicitáte, Perpétua, Agatha, Lúcia, Agnéte, Cæcília, Anastásia, et ómnibus Sanctis tuis: intra quorum nos consórtium, non æstimátor mériti, sed véniæ, quǽsumus, largítor admítte. Per Christum, Dóminum nostrum.
}{
À nous aussi, pécheurs, vos serviteurs, qui espérons en l’abondance de vos miséricordes, daignez accorder quelque participation à la société de vos saints apôtres et martyrs, avec Jean, le Baptiste, Étienne , Mathias , Barnabé , Ignace , Alexandre , Marcellin, Pierre , Félicité , Perpétue , Agathe, Lucie , Agnès , Cécile , Anastasie et avec tous vos saints ; vous qui donnez largement et ne regardez pas au mérite, mais au pardon, nous vous en prions, admettez-nous dans leur compagnie. Par le Christ notre Seigneur. Ainsi soit-il.
}

\twocoltext{
Per quem hæc ómnia, Dómine, semper bona creas, sanctí\cc{}ficas, viví\cc{}ficas, bene\cc{}dícis et præstas nobis.
Per ip\cc{}sum, et cum ip\cc{}so, et in ip\cc{}so, est tibi Deo Patri \cc omnipoténti, in unitáte Spíritus \cc Sancti,
omnis honor, et glória.
Per ómnia sǽcula sæculórum. \\
\rr Amen.
}{
Par lui, Seigneur, vous ne cessez de créer tous ces biens, de les sanctifier, de les vivifier, de les bénir et de nous les donner.
Par lui, et avec lui, et en lui, est à vous, Dieu le Père tout-puissant, en l’unité du Saint Esprit,
tout honneur et toute gloire.
Pour les siècles des siècles.\\
\rr Ainsi soit-il.
}

\twocoltext{
Orémus. \\
Præcéptis salutáribus móniti, et divína institutióne formáti audémus dícere:

Pater noster, qui es in cælis. Sanctificétur nomen tuum. Advéniat regnum tuum. Fiat volúntas tua, sicut in cælo et in terra. Panem nostrum quotidiánum da nobis hódie. Et dimítte nobis débita nostra, sicut et nos dimíttimus debitóribus nostris. Et ne nos indúcas in tentatiónem:\\
\rr Sed líbera nos a malo.\\
\vv Amen.
}{
Prions. Éclairés par de salutaires prescriptions et formés par l’enseignement divin, nous osons dire :

Notre Père, qui êtes aux cieux, que votre Nom soit sanctifié, que votre règne arrive, que votre volonté soit faite sur la terre comme au ciel. Donnez-nous aujourd’hui notre pain de chaque jour, pardonnez-nous nos offenses, comme nous pardonnons à ceux qui nous ont offensés, et ne nous laissez entrer en tentation. \\
\rr Mais délivrez-nous du mal.\\
\vv Amen.
}

\twocoltext{
Líbera nos, quǽsumus, Dómine, ab ómnibus malis, prætéritis, præséntibus et futúris: et intercedénte beáta et gloriósa semper Vírgine Dei Genetríce María, cum beátis Apóstolis tuis Petro et Paulo, atque Andréa, et ómnibus Sanctis, da propítius pacem in diébus nostris: ut, ope misericórdiæ tuæ adiúti, et a peccáto simus semper líberi et ab omni perturbatióne secúri.
Per eúndem Dóminum nostrum Iesum Christum, Fílium tuum.
Qui tecum vivit et regnat in unitáte Spíritus Sancti Deus.\\
\vv Per ómnia sǽcula sæculórum.\\
\rr Amen.\\
\vv Pax Dómini sit semper vobíscum.\\
\rr Et cum spíritu tuo.
}{
Délivrez-nous, Seigneur, nous vous en prions, de tous les maux passés, présents et à venir ; et par l’intercession de la bienheureuse et glorieuse Marie toujours vierge, Mère de Dieu, avec vos bienheureux apôtres Pierre et Paul, André, et tous les saints, soyez-nous favorable et donnez la paix à notre temps, afin qu’aidés par votre abondante miséricorde, nous soyons à jamais libérés du péché et préservés de toutes sortes de troubles.
Par le même Jésus-Christ, votre Fils, notre Seigneur,
qui, étant Dieu, vit et règne avec vous dans l’unité du Saint-Esprit.\\
\vv Dans tous les siècles des siècles.\\
\rr Ainsi soit-il.\\
\vv Que la paix du Seigneur soit toujours avec vous.\\
\rr Et avec votre esprit.
}

\twocoltext{
Hæc commíxtio, et consecrátio Córporis et Sánguinis Dómini nostri Iesu Christi, fiat accipiéntibus nobis in vitam ætérnam. Amen.
}{
Que ce mélange sacramentel du corps et du sang de notre Seigneur Jésus-Christ, que nous allons recevoir, nous serve pour la vie éternelle. Ainsi soit-il.
}

\gscore{ky_A10}{Agneau de Dieu, qui enlevez les péchés du monde : ayez pitié de nous.\\
Agneau de Dieu, qui enlevez les péchés du monde : ayez pitié de nous.\\
Agneau de Dieu, qui enlevez les péchés du monde : donnez-nous la paix.}

\smalltitle{Communion}

\twocoltext{
Dómine Iesu Christe, qui dixísti Apóstolis tuis: Pacem relínquo vobis, pacem meam do vobis: ne respícias peccáta mea, sed fidem Ecclésiæ tuæ; eámque secúndum voluntátem tuam pacificáre et coadunáre dignéris: Qui vivis et regnas Deus per ómnia sǽcula sæculórum. Amen.
}{
Seigneur Jésus-Christ, qui avez dit à vos apôtres : Je vous laisse la paix, Je vous donne ma paix, ne regardez pas mes péchés, mais la foi de votre Église ; et daignez, conformément à votre volonté, lui donner la paix et l’unité. Vous qui, étant Dieu, vivez et régnez dans tous les siècles des siècles. Ainsi soit-il.
}

\twocoltext{
Dómine Iesu Christe, Fili Dei vivi, qui ex voluntáte Patris, cooperánte Spíritu Sancto, per mortem tuam mundum vivificásti: líbera me per hoc sacrosánctum Corpus et Sánguinem tuum ab ómnibus iniquitátibus meis, et univérsis malis: et fac me tuis semper inhærére mandátis, et a te numquam separári permíttas: Qui cum eódem Deo Patre et Spíritu Sancto vivis et regnas Deus in sǽcula sæculórum. Amen.
}{
Seigneur Jésus-Christ, Fils du Dieu vivant, qui, selon la volonté du Père et avec la coopération de l’Esprit Saint, avez donné la vie au monde par votre mort ; libérez-moi par votre corps et votre sang sacrés de tous mes péchés et de tous les maux : faites que je m’attache toujours à vos commandements, et ne permettez pas que je sois jamais séparé de vous. Vous qui, étant Dieu, vivez et régnez avec le même Dieu le Père et le Saint-Esprit, dans les siècles des siècles. Ainsi soit-il.
}

\twocoltext{
Percéptio Córporis tui, Dómine Iesu Christe, quod ego indígnus súmere præsúmo, non mihi provéniat in iudícium et condemnatiónem: sed pro tua pietáte prosit mihi ad tutaméntum mentis et córporis, et ad medélam percipiéndam: Qui vivis et regnas cum Deo Patre in unitáte Spíritus Sancti Deus, per ómnia sǽcula sæculórum. Amen.
}{
Que la réception de votre corps, que j’ose prendre, tout indigne que je suis, Seigneur Jésus-Christ, n’entraîne pour moi ni jugement ni condamnation ; mais que, par votre bonté, elle serve de soutien et de remède à mon âme et à mon corps. Vous qui, étant Dieu, vivez et régnez avec Dieu le Père dans l’unité du Saint-Esprit, dans tous les siècles des siècles. Ainsi soit-il.
}

\twocoltext{
Panem cæléstem accípiam, et nomen Dómini invocábo.
Dómine, non sum dignus, ut intres sub tectum meum: sed tantum dic verbo, et sanábitur ánima mea.
}{
Je prendrai le Pain du ciel, et j’invoquerai le Nom du Seigneur.
Seigneur, je ne suis pas digne que vous entriez sous mon toit, mais dites seulement une parole et mon âme sera guérie.
}

\twocoltext{
\rubric{Le prêtre communie au Corps :}\\
Corpus Dómini nostri Iesu Christi custódiat ánimam meam in vitam ætérnam. Amen.
}{
~\\
Que le corps de notre Seigneur Jésus-Christ garde mon âme pour la vie éternelle. Ainsi soit-il.
}

\twocoltext{
Quid retríbuam Dómino pro ómnibus, quæ retríbuit mihi? Cálicem salutáris accípiam, et nomen Dómini invocábo. Laudans invocábo Dóminum, et ab inimícis meis salvus ero.\\
\rubric{Le prêtre communie au Sang :}\\
Sanguis Dómini nostri Iesu Christi custódiat ánimam meam in vitam ætérnam. Amen.
}{
Que rendrai-je au Seigneur pour tous ses bienfaits à mon égard ? Je prendrai le calice du salut et j’invoquerai le Nom du Seigneur. J’invoquerai le Nom du Seigneur en le louant, et je serai sauvé de mes ennemis.
~\\
Que le sang de notre Seigneur Jésus-Christ garde mon âme pour la vie éternelle. Ainsi soit-il.
}

\twocoltext{
Quod ore súmpsimus, Dómine, pura mente capiámus: et de múnere temporáli fiat nobis remédium sempitérnum.
Corpus tuum, Dómine, quod sumpsi, et Sanguis, quem potávi, adhǽreat viscéribus meis: et præsta; ut in me non remáneat scélerum mácula, quem pura et sancta refecérunt sacraménta: Qui vivis et regnas in sǽcula sæculórum. Amen.
}{
Ce que nous avons reçu par la bouche, Seigneur, que nous l’embrassions d’une âme pure, et que de ce don temporel nous vienne un remède éternel.
Que votre corps que j’ai pris et votre sang que j’ai bu, Seigneur, adhèrent à mes entrailles ; et faites que le péché ne laisse aucune tache en moi, que de purs et saints mystères ont restauré. Vous qui vivez et régnez dans les siècles des siècles. Ainsi soit-il.
}

\twocoltext{
\vv Ecce Agnus Dei, ecce qui tollit peccata mundi.\\
\rubric{On répond trois fois, en se frappant la poitrine :}\\
\rr Dómine, non sum dignus, ut intres sub tectum meum: sed tantum dic verbo, et sanábitur ánima mea.
}{
\vv Voici l'Agneau de Dieu, celui qui porte les péchés du monde.\\
~\\
\rr Seigneur, je ne suis pas digne que vous entriez sous mon toit, mais dites seulement une parole et mon âme sera guérie.
}

\gscore{co_beata_viscera}{\rubric{Ps. 44: 2, 5, 8} \aarub Heureuses les entrailles de la Vierge Marie, qui ont porté le Fils du Père éternel.\\
\vv \rubric{\emph{1. }} D'heureuses paroles jaillissent de mon cœur quand je dis mes poèmes pour le roi.\\
\vv \rubric{\emph{2. }} D'une langue aussi vive que la plume du scribe !\\
\vv \rubric{\emph{3. }} Dans ta beauté et ta bonté, marche, glorieuse, avance, et règne.\\
\vv \rubric{\emph{4. }} Pour la justice, la clémence et la vérité, ta main fait des merveilles.
}

\smalltitle{Postcommunion}

\twocoltext{
\vv Dóminus vobíscum.\\
\rr Et cum spíritu tuo.\\
\vv Orémus.\\
Sumptis, Dómine, salútis nostræ subsídiis: da, quǽsumus, beátæ Maríæ semper Vírginis patrocíniis nos ubíque prótegi; in cuius veneratióne hæc tuæ obtúlimus maiestáti.
Per Dóminum.\\
\rr Amen.

\rubric{Commémoration de saint Saturnin, martyr}

\vv Orémus.\\
Sanctíficet nos, quǽsumus, Dómine, tui percéptio sacraménti: et intercessióne Sanctórum tuórum tibi reddat accéptos.
Per Dóminum nostrum Iesum Christum, Fílium tuum: qui tecum vivit et regnat in unitáte Spíritus Sancti Deus, per ómnia sǽcula sæculórum.\\
\rr Amen.
}{
\vv Le Seigneur soit avec vous.\\
\rr Et avec votre esprit.\\
\vv Prions.\\
Ayant reçu ces moyens de salut, nous vous demandons, Seigneur, d'être toujours et partout assurés de la protecion de la bienheureuse Vierge Marie, en l'honneur de qui nous avons offert ce sacrifice à votre Majesté.
Par notre Seigneur Jésus-Christ, votre Fils, qui, étant Dieu, vit et règne avec vous, en l’unité du Saint-Esprit, dans tous les siècles des siècles.\\
\rr Amen.

\vv Prions.\\
Faites, nous vous en supplions, Seigneur, que la réception de ce sacrement nous sanctifie, et que grâce à l’intercession de vos Saints, elle nous rende agréables à vos yeux. 
Par Jésus-Christ, ton Fils, notre Seigneur, qui vit et règne avec toi dans l’unité du Saint-Esprit, Dieu, pour les siècles des siècles.\\
\rr Ainsi soit-il.
}

\smalltitle{Envoi}

\twocoltext{
\vv Dóminus vobíscum.\\
\rr Et cum spíritu tuo.
}{
\vv Le Seigneur soit avec vous.\\
\rr Et avec votre esprit.
}

\gscore{ky_I9}{\vvrub Allez, c'est l'envoi. \rrrub Nous rendons grâces à Dieu.}

\twocoltext{
Pláceat tibi, sancta Trínitas, obséquium servitútis meæ: et præsta; ut sacrifícium, quod óculis tuæ maiestátis indígnus óbtuli, tibi sit acceptábile, mihíque et ómnibus, pro quibus illud óbtuli, sit, te miseránte, propitiábile. Per Christum, Dóminum nostrum. Amen.
}{
Agréez, Trinité Sainte, l’hommage de mon ministère : et faites que le sacrifice que, malgré mon indignité, j’ai présenté aux regards de votre Majesté, vous soit agréable, et que, par votre miséricorde, il puisse attirer votre faveur sur moi et sur tous ceux pour lesquels je vous l’ai offert. Par le Christ notre Seigneur. Ainsi soit-il.
}

\twocoltext{
Benedícat vos omnípotens Deus,
Pater, et Fílius, \cc et Spíritus Sanctus. \\
\rr Amen.
}{
Que le Dieu tout-puissant vous bénisse,
le Père, le Fils, et le Saint Esprit.\\
\rr Ainsi soit-il.
}

\smalltitle{Dernier Évangile}

\twocoltext{
\vv Dóminus vobíscum.\\
\rr Et cum spíritu tuo.\\
\vv Inítium \cc sancti Evangélii secúndum Ioánnem\\
\rr Glória tibi, Dómine.}{\vv Le Seigneur soit avec vous.\\
\rr Et avec votre esprit.\\
\vv Commencement du saint Évangile selon saint Jean.\\
\rr Gloire à vous, Seigneur.}

\newpage

\twocoltext{\rubric{Jn. 1, 1-14}\\
In princípio erat Verbum, et Verbum erat apud Deum, et Deus erat Verbum. Hoc erat in princípio apud Deum. Omnia per ipsum facta sunt: et sine ipso factum est nihil, quod factum est: in ipso vita erat, et vita erat lux hóminum: et lux in ténebris lucet, et ténebræ eam non comprehendérunt.
Fuit homo missus a Deo, cui nomen erat Ioánnes. Hic venit in testimónium, ut testimónium perhibéret de lúmine, ut omnes créderent per illum. Non erat ille lux, sed ut testimónium perhibéret de lúmine.
Erat lux vera, quæ illúminat omnem hóminem veniéntem in hunc mundum. In mundo erat, et mundus per ipsum factus est, et mundus eum non cognóvit. In própria venit, et sui eum non recepérunt. Quotquot autem recepérunt eum, dedit eis potestátem fílios Dei fíeri, his, qui credunt in nómine eius: qui non ex sanguínibus, neque ex voluntáte carnis, neque ex voluntáte viri, sed ex Deo nati sunt. Genuflectit dicens: Et Verbum caro factum est, Et surgens prosequitur: et habitávit in nobis: et vídimus glóriam eius, glóriam quasi Unigéniti a Patre, plenum grátiæ et
veritátis.\\
\rr Deo grátias.
}{
Au commencement était le Verbe, et le Verbe était auprès de Dieu et le Verbe était Dieu. Il était au commencement auprès de Dieu. Toutes choses ont été faites par lui, et rien de ce qui a été fait n’a été fait sans lui. En lui était la vie, et la vie était la lumière des hommes ; et la lumière luit dans les ténèbres, et les ténèbres ne l’ont point comprise.
Il y eut un homme, envoyé de Dieu, appelé Jean. Il vint en témoin pour rendre témoignage à la lumière, afin que tous crussent par lui. Il n’était pas lui-même la lumière, mais il vint pour rendre témoignage à la lumière.
Celui-là était la vraie lumière qui éclaire tout homme venant en ce monde. Il était dans le monde, et le monde a été fait par lui, et le monde ne l’a pas reconnu. Il est venu chez lui, et les siens ne l’ont pas reçu.
Mais à tous ceux qui l’ont reçu, il a donné le pouvoir de devenir enfants de Dieu, à ceux qui croient en son nom : qui ne sont point nés du sang, ni de la volonté de la chair, ni de la volonté de l’homme, mais de Dieu. On fléchit le genou avec le prêtre, qui dit : Et le Verbe s’est fait chair, Et se relevant, le prêtre poursuit : et il a habité parmi nous, et nous avons vu sa gloire, qui est la gloire du Fils unique du Père, plein de grâce et de vérité.\\
\rr Nous rendons grâces à Dieu.
}

\end{document}