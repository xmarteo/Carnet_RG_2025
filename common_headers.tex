%KDP 6inx9in sans fonds perdus
\usepackage[paperwidth=6in, paperheight=9in]{geometry}

\usepackage{fontspec}
\usepackage[nolocalmarks]{polyglossia}
\usepackage[table]{xcolor}
\usepackage{fancyhdr}
\usepackage{titlesec}
\usepackage{setspace}
\usepackage{expl3}
\usepackage{hyperref}
\usepackage{refcount}
\usepackage{needspace}
\usepackage{etoolbox}
\usepackage{enumitem}
\usepackage{lettrine}
\usepackage{longtable}
\usepackage{luacode}
\usepackage{alltt}

\setdefaultlanguage[variant=french, frenchitemlabels=false]{french}

\geometry{
inner=25mm,
outer=14mm,
top=14mm,
bottom=14mm,
headsep=3mm,
}

%% General scale of all graphical elements.
%% Values different from 1 are largely untested.
%% Used in those commands (e.g. everything FontSpec) that use a scale parameter.
\newcommand{\customscale}{1}

%% Provide the command \fpevalc as a copy of the code-level \fp_eval:n.
%% \fpevalc allows to evaluate floating point calculation for scaled parameters, e.g. \setSomeStretchFactor{\fpevalc{0,6 * \customscale}}
\ExplSyntaxOn
\cs_new_eq:NN \fpevalc \fp_eval:n
\ExplSyntaxOff

%% No indentation of paragraphs
\setlength{\parindent}{0mm}

%% We want to allow large inter-words space 
%% to avoid overfull boxes in two-columns rubrics.
\sloppy


%%%%%%%%%%%%%%% GREGORIO CONFIG %%%%%%%%%%%%%%%

\usepackage[forcecompile]{gregoriotex}

%% this limits how much scores can stretch vertically
%% when they are forced to adhere to the bottom of a page (i.e followed by \pagebreak)
%\grechangedim{baselineskip}{65pt plus 1pt minus 5pt}{scalable}

%% text above lines shall be of color gregoriocolor
\grechangestyle{abovelinestext}{\color{gregoriocolor}\footnotesize\itshape}
%% fine-tuning of space beween the staff and the text above lines
\newcommand{\altraise}{0mm} %% default is -0.1cm
\grechangedim{abovelinestextraise}{\altraise}{scalable}
\grechangedim{abovelinestextheight}{10mm}{scalable}

%% fine-tuning of space between the staff and the lyrics
\newcommand{\textraise}{2.8ex} %% default is 3.48471 ex
\grechangedim{spacelinestext}{\textraise}{scalable}

%% fine-tuning of space between the initial and the annotations
\newcommand{\annraise}{0mm} %% default is -0.2mm
\grechangedim{annotationraise}{\annraise}{scalable}

%% fine-tuning the behavior of text placed under bars. We use the so-called "new algorithm" which
%% places the bar in the middle of surrounding notes, and the text in the middle of surrounding text.
%% however, we restrict drastically the deviation of the text from the position of the bar.
\grechangedim{maxbaroffsettextleft}{0.5mm}{scalable}
\grechangedim{maxbaroffsettextright}{0.5mm}{scalable}

%% in case we show NABC, font selection
\gresetnabcfont{gregall}{10} 

\newcommand{\incipit}[2]{
  \gresetinitiallines{0}
  \gregorioscore{\subfix{incipits/#1_#2}}
  \gresetinitiallines{1}
}

% we want to print NABC 'cause this is the RG
\gresetnabc{1}{visible}

%correction of the space before and after the score by exchanging spaceabovelines for spacebeneathtext
\grechangedim{spaceabovelines}{-3mm}{scalable}
\grechangedim{spacebeneathtext}{3mm}{scalable}

%%%%%%%%%%%%%%% FONTS %%%%%%%%%%%%%%%

%%%%%%%%%%%%%%% Main font
\setmainfont[Ligatures=TeX, Scale=\customscale]{Charis}
\setstretch{\fpevalc{1.05 * \customscale}}

%%%%%%%%%%%%%%% Score initials
%% \initialsize resizes the initials, with one argument (size in points)
\newcommand{\initialsize}[1]{
    \grechangestyle{initial}{\fontspec{Zallman Caps}\fontsize{#1}{#1}\selectfont}
}
%% default initial size is 32 points
\newcommand{\defaultinitialsize}{28}
\initialsize{\defaultinitialsize}

%% spacing before and after initials to kern the Zallman Caps.
%% this should be changed if we move away from Zallman Caps.
\newcommand{\initialspace}[2]{
  \grechangedim{afterinitialshift}{#2}{scalable}
  \grechangedim{beforeinitialshift}{#1}{scalable}
}
%% default space before and after initials is 0mm to the left and 2mm to the right.
\newcommand{\defaultannotationshift}{-2mm}
\newcommand{\defaultinitialspace}{\initialspace{0mm}{-\defaultannotationshift}}
\defaultinitialspace{}

%%%%%%%%%%%%%%% Score annotations

%% This global variable contains the office-part of the piece currently being
%% typeset, for indexing and annotation purposes.
\newcommand{\officepartbuffer}{}
%% This global variable contains the file name, i.e. label,
%% of the piece currently being typeset.
\newcommand{\labelbuffer}{}
%% This global variable contains the name of the index (A, H, R) relevant to the
%% piece currently being typeset, if any indexing is needed for that piece.
\newcommand{\indexnamebuffer}{}
%% This global variable contains the order number of the piece in the current hour
%% for annotation purposes ("Ant. 1", "Ant. 2", etc.)
\newcommand{\numberbuffer}{}
%% This global variable contains the mode+differentia of the last antiphon
%% for the purposes of fetching the correct pointed psalm
\newcommand{\nextpsalmmode}{}
%% This global variable contains the mode+differentia of the last piece
%% for indexing and annotation purposes
\newcommand{\indexedmode}{}

%% Given an office-part header (from Gregorio header capture), 
%% this section annotates the score according to the office-part, 
%% stored in \officepartbuffer, and the number stored by \gscore in \numberbuffer
%% and stores into \indexnamebuffer the corresponding index, for later use.
\newcommand{\abbrev}[4]{
  \IfSubStr{#1}{#2}{%
	\renewcommand{\officepartbuffer}{#3}%
	\renewcommand{\indexnamebuffer}{#4}%
  }%
  {}%
}
\newcommand{\officepartannotation}[1]{%
  \renewcommand{\officepartbuffer}{#1}%
  \renewcommand{\indexnamebuffer}{}%
  \abbrev{#1}{Antiphona}{Ant.}{A}%
  \abbrev{#1}{Hymnus}{Hy.}{H}%
  \abbrev{#1}{ntro}{Intr.}{I}%
  \abbrev{#1}{re}{Resp.}{R}%
  \abbrev{#1}{espo}{Resp.}{R}%
  \abbrev{#1}{adu}{Gr.}{G}%
  \abbrev{#1}{ll}{All.}{Al}%
  \abbrev{#1}{act}{Tract.}{T}%
  \abbrev{#1}{equen}{Seq.}{}%
  \abbrev{#1}{ffert}{Off.}{O}%
  \abbrev{#1}{ommun}{Co.}{C}%
  \abbrev{#1}{an}{Ant.}{A}%
  \abbrev{#1}{ntiph}{Ant.}{A}%
  \abbrev{#1}{ntic}{Cant.}{}%
  \abbrev{#1}{ymn}{Hy.}{H}%
  \abbrev{#1}{salm}{}{}%
  \abbrev{#1}{yrial}{}{}%
  \abbrev{#1}{Responsorium breve}{R.Br.}{R}%
  \abbrev{#1}{Toni Communes}{}{}%
  \greannotation{\officepartbuffer \numberbuffer}%
}

%%%% This is deep magic required to define \grenewcommand, which is the same as
%%%% \renewcommand but has global effects
\makeatletter
\def\gnewcommand{\g@star@or@long\new@command}
\def\grenewcommand{\g@star@or@long\renew@command}
\def\g@star@or@long#1{% 
  \@ifstar{\let\l@ngrel@x\global#1}{\def\l@ngrel@x{\long\global}#1}}
\makeatother
%%%% end deep magic

\newcommand{\changemode}[3]{%
  \ifstrequal{#1}{#2}{%
	\grenewcommand{\nextpsalmmode}{#3}%
  }%
  {}%
}
\newcommand{\modeannotation}[1]{
  \grenewcommand{\nextpsalmmode}{#1}
  \renewcommand{\indexedmode}{#1}	
  \changemode{#1}{1a}{1a}
  \changemode{#1}{1a2}{1a2}
  \changemode{#1}{1a3}{1a3}
  \changemode{#1}{1f}{1f}
  \changemode{#1}{1g}{1g}
  \changemode{#1}{1g2}{1g2}
  \changemode{#1}{1g3}{1g3}
  \changemode{#1}{2D}{2}
  \changemode{#1}{3a}{3a}
  \changemode{#1}{3b}{3b}
  \changemode{#1}{3g}{3g}
  \changemode{#1}{3g2}{3g2}
  \changemode{#1}{4c}{4-alt-c}
  \changemode{#1}{4E}{4E}
  \changemode{#1}{4g}{4g}
  \changemode{#1}{5a}{5}
  \changemode{#1}{6F}{6}
  \changemode{#1}{7a}{7a}
  \changemode{#1}{7b}{7b}
  \changemode{#1}{7c}{7c}
  \changemode{#1}{7c2}{7c2}
  \changemode{#1}{8c}{8c}
  \changemode{#1}{8G}{8G}
  \changemode{#1}{Per.}{per}
  \greannotation{\indexedmode}
}

%% Given a name header (from Gregorio header capture),
%% this section labels the piece with the piece label and label name,
%% and indexes the piece accordint to its office-part and mode
\begin{luacode*}
function delete_newline ( s )
   s = string.gsub ( s, 'newline', '')
   s = string.gsub ( s, 'hspace', '')
   s = string.gsub ( s, '4mm', '')
   s = string.gsub ( s, 'protect', '')
   s = string.gsub ( s, 'hbox', '')
   s = string.gsub ( s, [[\]], '')
   s = string.gsub ( s, "{}", '')
   tex.sprint ( s )
end
\end{luacode*}
\makeatletter
\newcommand{\namecapture}[1]{%
  %% this prevents page breaks between the phantom section and its label, and the actual score.
  \needspace{4\baselineskip}%
  \protected@edef\@currentlabelname{\directlua{delete_newline(\luastring{#1})}}%
  \phantomsection%
  \label{\labelbuffer}%
  %% indexation
  \ifdefstrequal{H}{\indexnamebuffer}{%
    \index[\indexnamebuffer]{{#1}@\indexedmode & #1}%
  }{}%
  \ifdefstrequal{A}{\indexnamebuffer}{%
    \index[\indexnamebuffer]{{#1}@\indexedmode & #1}%
  }{}%
  \ifdefstrequal{R}{\indexnamebuffer}{%
    \index[\indexnamebuffer]{#1}%
  }{}%
}
\makeatother

\gresetheadercapture{office-part}{officepartannotation}{}
\gresetheadercapture{mode}{modeannotation}{string}
\gresetheadercapture{name}{namecapture}{}

\newcommand{\smallscore}[2]{
  \gresetinitiallines{0}
  \gregorioscore{\subfix{gabc/#1}}
  \gresetinitiallines{1}
  \translation{#2}
}

\newcommand{\gscore}[3][]{
  %% #1 : (optional) the number to be printed near "Ant." in a multi-antiphon hour
  %% #2 : the hudelmaier code, which is also the name of the file, and the label.
  %% #3 : the translation of the piece.
  \renewcommand{\labelbuffer}{#2}
  \renewcommand{\numberbuffer}{#1}
  \gregorioscore{gabc/#2}
  \translation{#3}
}

\newcommand{\hymnus}[1]{
  %% #1 : the hudelmaier code, which is also the name of the file,
  %%      and the label, and the name of the translation file
  \renewcommand{\labelbuffer}{#1}
  \renewcommand{\numberbuffer}{}
  \columnratio{0.75}
  \begin{paracol}{2}
  \gregorioscore{gabc/#1}
  \switchcolumn
  %\begin{alltt}\normaltext%
  \input{hymni_fr/#1}
  %\end{alltt}
  \end{paracol}
  \columnratio{0.5}
}

%%%%%%%%%%%%%%% GRAPHICAL ELEMENTS %%%%%%%%%%%%%%%

%% V/, R/, A/ and + signs for in-line use (\vv \rr \aa \cc)
\newcommand{\specialcharhsep}{3mm} % space after invoking R/ or V/ or A/ outside rubrics
\newcommand{\vv}{\textcolor{gregoriocolor}{\fontspec[Scale=\customscale]{Charis}℣.\nolinebreak[4]\hspace{\specialcharhsep}\nolinebreak[4]}}
\newcommand{\rr}{\textcolor{gregoriocolor}{\fontspec[Scale=\customscale]{Charis}℟.\nolinebreak[4]\hspace{\specialcharhsep}\nolinebreak[4]}}
\renewcommand{\aa}{\textcolor{gregoriocolor}{\fontspec[Scale=\customscale]{Charis}\Abar.\nolinebreak[4]\hspace{\specialcharhsep}\nolinebreak[4]}}
\newcommand{\cc}{\textcolor{gregoriocolor}{\fontspec[Scale=\customscale]{FreeSerif}\symbol{"2720}~}}
%% Same special characters, for in-score use (<sp>V/ R/ A/ +</sp>)
\gresetspecial{V/}{\textcolor{gregoriocolor}{\fontspec[Scale=\customscale]{Charis}℣.~}}
\gresetspecial{R/}{\textcolor{gregoriocolor}{\fontspec[Scale=\customscale]{Charis}℟.~}}
\gresetspecial{A/}{\textcolor{gregoriocolor}{\fontspec[Scale=\customscale]{Charis}\Abar.~}}
\gresetspecial{+}{{\fontspec[Scale=\customscale]{FreeSerif}†~}}
\gresetspecial{*}{\gresixstar}
\gresetspecial{cross}{\textcolor{gregoriocolor}{\fontspec[Scale=\customscale]{FreeSerif}\symbol{"2720}}}
\gresetspecial{labiacross}{\textcolor{gregoriocolor}{+}}
%% Same special characters, for use in rubrics (no space, and no red command since it will be reddified with the rest)
\newcommand{\vvrub}{{\fontspec[Scale=\customscale]{Charis}℣.~}}
\newcommand{\rrrub}{{\fontspec[Scale=\customscale]{Charis}℟.~}}
\newcommand{\aarub}{{\fontspec[Scale=\customscale]{Charis}\Abar.~}}

%% the asterisk as found in the mediants of text-only psalms
\newcommand{\psstar}{\GreSpecial{*}}
\newcommand{\pscross}{\GreSpecial{+}}
%% also, most psalms do not call those but use † and * - todo

%% Macro to print rubrics and normal text inside of them
\newcommand{\rubric}[1]{\textcolor{gregoriocolor}{\emph{#1}}}
\newcommand{\normaltext}[1]{{\normalfont\normalcolor #1}}

%% Macro to print the name of a score in normal characters inside a \rubric
\newcommand{\scorename}[1]{\normaltext{\nameref{M-#1}}}

%% Macro to print initials of some texts in red
\newcommand{\colored}[1]{
	\textcolor{gregoriocolor}{#1}%
}

%% Macro to print rubrics that inaugurate a section of more rubrics (in red, normal font, centered)
\newcommand{\titlerubric}[1]{{\centering{}\textcolor{gregoriocolor}{#1}\par}}

%% Macro to print french-language inflexions as underlined
\newcommand{\frflex}[1]{\underline{#1}}

%% Macros to print preparatory and accented syllables in psalms
\newcommand{\sylac}[1]{\textbf{#1}}
\newcommand{\sylprep}[1]{\textit{#1}}

%% Macro to print the blessing 
\newcommand{\blessing}{
	\vspace{2mm}
	\versiculus{Dóminus vobíscum.}{Et cum spíritu tuo.}{Le Seigneur soit avec vous.}{Et avec votre esprit.}
	\versiculus{Benedícat vos omnípotens Deus, \cc Pater, et Fílius, et Spíritus Sanctus.}{Amen.}{Et que Dieu tout-puissant vous bénisse, le Père, le Fils, et le Saint-Esprit.}{Amen.}
}

%% Macro to print the Pater secreto
\newcommand{\patersecreto}{
	\versiculus{Pater noster... \rubric{secreto usque ad} Et ne nos indúcas in tentatiónem:}{Sed líbera nos a malo.}{Notre Père... \rubric{en silence jusqu'à} et ne nous laisse pas entrer en tentation.}{Mais délivre-nous du mal.}
}

\newcommand{\absolutio}[2]{
	\twocoltext{
		\rubric{Absolutio.} #1 \\
		\rr Amen.
	}{
		\rubric{Absolution.} #2 \\
		\rr Amen.
	}
}

\newcommand{\benedictio}[2]{
	\versiculus{Iube, domne, benedícere.\\\rubric{Benedictio.} #1}{Amen.}{Veuillez, maître, bénir.\\\rubric{Bénédiction} #2}{Amen.}
}

\newcommand{\tuautem}{
	\versiculus{Tu autem, Dómine, miserére nobis.}{Deo grátias.}{Et toi Seigneur, aie pitié de nous.}{Nous rendons grâces à Dieu.}
}

\newcommand{\fideliumanimae}{
	\versiculus{Fidélium ánimæ per misericórdiam Dei requiéscant in pace.}{Amen.}{Que par la miséricorde de Dieu, les âmes des fidèles trépassés reposent en paix.}{Amen.}
}

\newcommand{\dominusvobiscum}{
	\versiculus{Dóminus vobíscum.}{Et cum spíritu tuo.}{Le Seigneur soit avec vous.}{Et avec votre esprit.}
}


%%%%%%%%%%%%%%% PAGE HEADERS AND FOOTERS %%%%%%%

\pagestyle{fancy}
\fancyhead{}
\fancyfoot{}
\renewcommand{\headrulewidth}{0pt}
\setlength{\headheight}{20pt}
\fancyhead[RO]{\small\rightmark\hspace{1cm}\thepage}
\fancyhead[LE]{\small\thepage\hspace{1cm}\leftmark}

%%%%%%%%%%%%%%% COLUMN MANAGEMENT %%%%%%%%%%%%%%%

\usepackage{paracol}
\usepackage{multicol}
\setlength\columnseprule{0.4pt}
\setlength{\multicolsep}{6pt plus 2pt minus 1.5pt}

\newcommand{\twocolitemizedtext}[2]{
	%% macro générique pour les cantiques et psaumes
	%% #1 premier fichier
	%% #2 deuxième fichier
	\begin{paracol}{2}
	\begin{itemize}[
		label=\null, 
		leftmargin=0pt, 
		itemindent=10pt, 
		labelsep=0pt, 
		labelwidth=0pt, 
		rightmargin=0pt, 
		parsep=0pt, 
		itemsep=0pt,
		topsep=-2mm]
	\input{#1}
	\end{itemize}
	\switchcolumn
	\begin{itemize}[
		label=\null, 
		leftmargin=0pt, 
		itemindent=10pt, 
		labelsep=0pt, 
		labelwidth=0pt, 
		rightmargin=0pt, 
		parsep=0pt, 
		itemsep=0pt,
		topsep=-2mm]
	\input{#2}
	\end{itemize}
	\end{paracol}
}

\newcommand{\psalmVulgate}[2][\nextpsalmmode]{
	%% #1 (optionnel) : mode+differentia. par défaut: mode de la dernière pièce gabc parsée.
	%% #2 numéro du psaume
	\smalltitle{Psaume #2}
	\gresetinitiallines{0}
	\gregorioscore{\subfix{incipits_vulgata/#2-#1}}
	\gresetinitiallines{1}
	\twocolitemizedtext{psalmi_vulgata/#2-#1.tex}{psalmi_fr/#2.tex}
}

\newcommand{\psalmNova}[2][\nextpsalmmode]{
	%% #1 (optionnel) : mode+differentia. par défaut: mode de la dernière pièce gabc parsée.
	%% #2 numéro du psaume
	\smalltitle{Psaume #2}
	\gresetinitiallines{0}
	\gregorioscore{\subfix{incipits_nova/#2-#1}}
	\gresetinitiallines{1}
	\twocolitemizedtext{psalmi_nova/#2-#1.tex}{psalmi_fr/#2.tex}
}

\newcommand{\canticumVulgate}[3][\nextpsalmmode]{
	%% #1 (optionnel) : mode+differentia. par défaut: mode de la dernière pièce gabc parsée.
	%% #2 nom de fichier associé au cantique
	%% #3 titre du cantique
	\smalltitle{Cantique #3}
	\gresetinitiallines{0}
	\gregorioscore{\subfix{incipits_vulgata/#2-#1}}
	\gresetinitiallines{1}
	\twocolitemizedtext{psalmi_vulgata/#2-#1.tex}{psalmi_fr/#2.tex}
}

\newcommand{\canticumNova}[3][\nextpsalmmode]{
	%% #1 (optionnel) : mode+differentia. par défaut: mode de la dernière pièce gabc parsée.
	%% #2 nom de fichier associé au cantique
	%% #3 titre du cantique
	\smalltitle{Cantique #3}
	\gresetinitiallines{0}
	\gregorioscore{\subfix{incipits_nova/#2-#1}}
	\gresetinitiallines{1}
	\twocolitemizedtext{psalmi_nova/#2-#1.tex}{psalmi_fr/#2.tex}
}

\newcommand{\twocoltext}[2]{
	\begin{paracol}{2}
#1
	\switchcolumn
#2
	\end{paracol}
}

\newcommand{\oratio}[2]{
	\smalltitle{Oraison}
	\twocoltext{#1}{#2}
}

\newcommand{\versiculus}[4]{
	\twocoltext{
		\vv #1 \\ \rr #2
	}{
		\vv #3 \\ \rr #4
	}
}

\newcommand{\lectiobrevis}[3]{
\smalltitle{Lecture brève}
\twocoltext{\rubric{#1}\\#2}{#3}
}

\newcommand{\capitulum}[3]{
\smalltitle{Capitule}
\twocoltext{\rubric{#1}\\#2}{#3}
}

\newcommand{\columnbalancinglatin}[1]{
\vspace{2mm}#1
}

\newcommand{\smalltitle}[1]{
  \vspace{0.5\baselineskip}
  \par{\centering\textbf{#1}\par}
  \vspace{0.5\baselineskip}
}

\newcommand{\bigtitle}[3]{
  \begin{center}
	{\fontspec[Scale=\customscale]{Futura Book}
	\MakeUppercase{\Large #1}\\
	}
  \end{center}
  \renewcommand{\rightmark}{\hyphenpenalty=50{\sc#3}}
  \renewcommand{\leftmark}{\hyphenpenalty=50{\sc#2}}
  \thispagestyle{empty}
  \addcontentsline{toc}{section}{#3}
}

\newcommand{\nocturntitle}[1]{
  \begin{center}
	{\fontspec[Scale=\customscale]{Futura Book}
	\MakeUppercase{\Large #1}\\
	}
  \end{center}
  \addcontentsline{toc}{subsection}{#1}
}

 \newcommand{\translation}[1]{%
	\emph{#1}
 }
 
\makeatletter
\newcommand{\@printReading}[1]{
    \def\content{%
        {#1
        }
    \par\noindent}
    \content
\egroup\endgroup}
\newcommand\reading{\begingroup%
  \begingroup%
    \lccode`\~=9\relax%
    \lowercase{%
  \endgroup
    \def~%
  }{\hspace{8mm}}%
  \catcode9=\active%
  \bgroup\obeylines\@printReading}
\makeatother

\newcommand{\hymntranslation}[1]{
 	\begin{multicols}{2}
	\reading{#1}
	\end{multicols}
}
 
%%%%%%%%%%%%%%% SUBFILES %%%%%%%%%%%%%%%

\usepackage{xr}
\usepackage{subfiles}

%% When we start a new subfile (new chapter), 
%% we start on a new page (with blank filling on the previous page) and create a corresponding label.
\newcommand{\customsubfile}[1]{\newpage\label{#1}\thispagestyle{empty}\subfile{#1}}
