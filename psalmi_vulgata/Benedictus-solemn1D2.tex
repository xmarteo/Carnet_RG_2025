\item Benedíctus Dómi\sylprep{nus}, \sylprep{De}\sylprep{us} \sylac{Is}\sylac{ra}ël:~* quia visitávit, et fecit redemptiónem \sylprep{ple}\sylprep{bis} \sylac{su}æ:
\item Et eréxit cornu \sylprep{sa}\sylprep{lú}\sylprep{tis} \sylac{no}bis:~* in domo David, pú\sylprep{e}\sylprep{ri} \sylac{su}i.
\item Sicut locútus est \sylprep{per} \sylprep{os} \sylprep{sanc}\sylac{tó}rum,~* qui a sǽculo sunt, prophe\sylprep{tá}\sylprep{rum} \sylac{e}jus:
\item Salútem ex in\sylprep{i}\sylprep{mí}\sylprep{cis} \sylac{nos}tris,~* et de manu ómnium, \sylprep{qui} \sylprep{o}\sylac{dé}\sylac{runt} nos.
\item Ad faciéndam misericórdiam cum \sylprep{pá}\sylprep{tri}\sylprep{bus} \sylac{nos}tris:~* et memorári testaménti \sylprep{su}\sylprep{i} \sylac{sanc}ti.
\item Jusjurándum, quod jurávit ad Abra\sylprep{ham} \sylprep{pa}\sylprep{trem} \sylac{nos}trum,~* datú\sylprep{rum} \sylprep{se} \sylac{no}bis:
\item Ut sine timóre, de manu inimicórum nostró\sylprep{rum} \sylprep{li}\sylprep{be}\sylac{rá}ti,~* servi\sylprep{á}\sylprep{mus} \sylac{il}li.
\item In sanctitáte, et justíti\sylprep{a} \sylprep{co}\sylprep{ram} \sylac{ip}so,~* ómnibus di\sylprep{é}\sylprep{bus} \sylac{nos}tris.
\item Et tu, puer, Prophéta Altís\sylprep{si}\sylprep{mi} \sylprep{vo}\sylac{cá}\sylac{be}ris:~* præíbis enim ante fáciem Dómini, paráre \sylprep{vi}\sylprep{as} \sylac{e}jus:
\item Ad dandam sciéntiam salú\sylprep{tis} \sylprep{ple}\sylprep{bi} \sylac{e}jus:~* in remissiónem peccató\sylprep{rum} \sylprep{e}\sylac{ó}rum:
\item Per víscera misericórdi\sylprep{æ} \sylprep{De}\sylprep{i} \sylac{nos}tri:~* in quibus visitávit nos, óri\sylprep{ens} \sylprep{ex} \sylac{al}to:
\item Illumináre his, qui in ténebris, et in um\sylprep{bra} \sylprep{mor}\sylprep{tis} \sylac{se}dent:~* ad dirigéndos pedes nostros in \sylprep{vi}\sylprep{am} \sylac{pa}cis.
\item Glória \sylprep{Pa}\sylprep{tri}, \sylprep{et} \sylac{Fí}\sylac{li}o,~* et Spirí\sylprep{tu}\sylprep{i} \sylac{Sanc}to.
\item Sicut erat in princípio, \sylprep{et} \sylprep{nunc}, \sylprep{et} \sylac{sem}per,~* et in sǽcula sæcu\sylprep{ló}\sylprep{rum}. \sylac{A}men.
