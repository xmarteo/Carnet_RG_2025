\item Seigneur, écoute la justice !\pscross{} Entends ma plainte, accu\frflex{e}ille ma prière :\psstar{} mes lèvres ne m\frflex{e}ntent pas.
\item De ta face, me viendr\frflex{a} la sentence :\psstar{} tes yeux verr\frflex{o}nt où est le droit.
\item Tu sondes mon cœur, tu me visites la nuit,\pscross{} tu m’éprouves, sans ri\frflex{e}n trouver ;\psstar{} mes pensées n’ont pas franch\frflex{i} mes lèvres.
\item Pour me conduire sel\frflex{o}n ta parole,\psstar{} j’ai gardé le chem\frflex{i}n prescrit ;
\item J’ai tenu mes p\frflex{a}s sur tes traces :\psstar{} jamais mon pi\frflex{e}d n’a trébuché.
\item Je t’appelle, toi, le Die\frflex{u} qui répond :\psstar{} écoute-moi, ent\frflex{e}nds ce que je dis.
\item Montre les merv\frflex{e}illes de ta grâce,\psstar{} toi qui libères de l’agresseur ceux qui se réfug\frflex{i}ent sous ta droite.
\item Garde-moi comme la prun\frflex{e}lle de l’œil ;\psstar{} à l’ombre de tes \frflex{a}iles, cache-moi,
\item Loin des méch\frflex{a}nts qui m’ont ruiné,\psstar{} des ennemis mort\frflex{e}ls qui m’entourent.
\item Ils s’enferment d\frflex{a}ns leur suffisance ;\psstar{} l’arrogance à la bo\frflex{u}che, ils parlent.
\item Ils sont sur mes pas : mainten\frflex{a}nt ils me cernent,\psstar{} l’œil sur moi, pour me jet\frflex{e}r à terre,
\item Comme des lions pr\frflex{ê}ts au carnage,\psstar{} de jeunes fauves tap\frflex{i}s en embuscade.
\item Lève-toi, Seigneur, affronte-l\frflex{e}s, renverse-les ;\psstar{} par ton épée, libère-m\frflex{o}i des méchants.
\item Que ta main, Seigneur, les excl\frflex{u}e d’entre les hommes,\psstar{} hors de l’humanité, hors de ce monde : tel soit le s\frflex{o}rt de leur vie !
\item Réserve-leur de quoi les rassasier :\pscross{} que leurs fils en s\frflex{o}ient saturés,\psstar{} qu’il en reste enc\frflex{o}re pour leurs enfants !
\item Et moi, par ta justice, je verr\frflex{a}i ta face :\psstar{} au réveil, je me rassasier\frflex{a}i de ton visage.