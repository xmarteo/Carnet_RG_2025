\item Criez de joie pour Die\frflex{u}, notre force,\psstar{} acclamez le Die\frflex{u} de Jacob.
\item Jouez, musiques, frapp\frflex{e}z le tambourin,\psstar{} la harpe et la cith\frflex{a}re mélodieuse.
\item Sonnez du cor pour le m\frflex{o}is nouveau,\psstar{} quand revient le jo\frflex{u}r de notre fête.
\item C’est là, pour Isra\frflex{ë}l, une règle,\psstar{} une ordonnance du Die\frflex{u} de Jacob ;
\item Il en fit, pour Jos\frflex{e}ph, une loi\psstar{} quand il marcha contre la t\frflex{e}rre d’Égypte.
\item J’entends des mots qui m’étaient inconnus :\pscross{} « J’ai ôté le poids qui charge\frflex{a}it ses épaules ;\psstar{} ses mains ont dépos\frflex{é} le fardeau.
\item Quand tu criais sous l’oppression, je t’ai sauvé ;\pscross{} je répondais, cach\frflex{é} dans l’orage,\psstar{} je t’éprouvais près des ea\frflex{u}x de Mériba.
\item Écoute, je t’adj\frflex{u}re, ô mon peuple ;\psstar{} vas-tu m’écout\frflex{e}r, Israël ?
\item Tu n’auras pas chez t\frflex{o}i d’autres dieux,\psstar{} tu ne serviras aucun die\frflex{u} étranger.
\item C’est moi, le Seigneur ton Dieu,\pscross{} qui t’ai fait monter de la t\frflex{e}rre d’Égypte !\psstar{} Ouvre ta bouche, m\frflex{o}i, je l’emplirai.
\item Mais mon peuple n’a pas écout\frflex{é} ma voix,\psstar{} Israël n’a pas voul\frflex{u} de moi.
\item Je l’ai livré à son cœ\frflex{u}r endurci :\psstar{} qu’il aille et su\frflex{i}ve ses vues !
\item Ah ! Si mon pe\frflex{u}ple m’écoutait,\psstar{} Israël, s’il all\frflex{a}it sur mes chemins !
\item Aussitôt j’humilier\frflex{a}is ses ennemis,\psstar{} contre ses oppresseurs je tourner\frflex{a}is ma main.
\item Mes adversaires s’abaisser\frflex{a}ient devant lui ;\psstar{} tel serait leur s\frflex{o}rt à jamais !
\item Je le nourrirais de la fle\frflex{u}r du froment,\psstar{} je te rassasierais avec le mi\frflex{e}l du rocher ! »