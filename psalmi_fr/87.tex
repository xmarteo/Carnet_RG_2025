\item Seigneur, mon Die\frflex{u} et mon salut,~\psstar{} dans cette nuit où je cr\frflex{i}e en ta présence,
\item Que ma prière parvi\frflex{e}nne jusqu’à toi,~\psstar{} ouvre l’or\frflex{e}ille à ma plainte.
\item Car mon âme est rassasi\frflex{é}e de malheur,~\psstar{} ma vie est au b\frflex{o}rd de l’abîme ;
\item On me voit déjà desc\frflex{e}ndre à la fosse,~\psstar{} je suis comme un h\frflex{o}mme fini.
\item Ma place est parm\frflex{i} les morts,~\psstar{} avec ceux que l’on a tu\frflex{é}s, enterrés,
\item Ceux dont tu n’as pl\frflex{u}s souvenir,~\psstar{} qui sont exclus, et l\frflex{o}in de ta main.
\item Tu m’as mis au plus prof\frflex{o}nd de la fosse,~\psstar{} en des lieux englout\frflex{i}s, ténébreux ;
\item Le poids de ta col\frflex{è}re m’écrase,~\psstar{} tu déverses tes fl\frflex{o}ts contre moi.
\item Tu éloignes de m\frflex{o}i mes amis,~\psstar{} tu m’as rendu abomin\frflex{a}ble pour eux ;
\item Enfermé, je n’ai p\frflex{a}s d’issue :~\psstar{} à force de souffrir, mes ye\frflex{u}x s’éteignent.
\item Je t’appelle, Seigne\frflex{u}r, tout le jour,~\psstar{} je tends les m\frflex{a}ins vers toi :
\item Fais-tu des mir\frflex{a}cles pour les morts ?~\psstar{} leur ombre se dresse-t-\frflex{e}lle pour t’acclamer ?
\item Qui parlera de ton amo\frflex{u}r dans la tombe,~\psstar{} de ta fidélité au roya\frflex{u}me de la mort ?
\item Connaît-on dans les tén\frflex{è}bres tes miracles,~\psstar{} et ta justice, au pa\frflex{y}s de l’oubli ?
\item Moi, je crie vers t\frflex{o}i, Seigneur ;~\psstar{} dès le matin, ma pri\frflex{è}re te cherche :
\item Pourquoi me rejet\frflex{e}r, Seigneur,~\psstar{} pourquoi me cach\frflex{e}r ta face ?
\item Malheureux, frappé à m\frflex{o}rt depuis l’enfance,~\psstar{} je n’en peux plus d’endur\frflex{e}r tes fléaux ;
\item Sur moi, ont déferl\frflex{é} tes orages :~\psstar{} tes effrois m’ont rédu\frflex{i}t au silence.
\item Ils me cernent comme l’ea\frflex{u} tout le jour,~\psstar{} ensemble ils se ref\frflex{e}rment sur moi.
\item Tu éloignes de moi am\frflex{i}s et familiers ;~\psstar{} ma compagne, c’\frflex{e}st la ténèbre.