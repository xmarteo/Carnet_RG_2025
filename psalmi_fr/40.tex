\item Heureux qui pense au pa\frflex{u}vre et au faible :~\psstar{} le Seigneur le sauve au jo\frflex{u}r du malheur !
\item Il le protège et le garde en vie, heure\frflex{u}x sur la terre.~\psstar{} Seigneur, ne le livre pas à la merc\frflex{i} de l’ennemi !
\item Le Seigneur le soutient sur son l\frflex{i}t de souffrance :~\psstar{} si malade qu’il s\frflex{o}it, tu le relèves.
\item J’avais dit : « Pitié pour m\frflex{o}i, Seigneur,~\psstar{} guéris-moi, car j’ai péch\frflex{é} contre toi ! »
\item Mes ennemis me cond\frflex{a}mnent déjà :~\psstar{} « Quand sera-t-il mort ? son n\frflex{o}m, effacé ? »
\item Si quelqu’un vient me voir, ses propos sont vides ;\pscross{} il emplit son cœ\frflex{u}r de pensées méchantes,~\psstar{} il sort, et dans la r\frflex{u}e il parle.
\item Unis contre moi, mes ennem\frflex{i}s murmurent,~\psstar{} à mon sujet, ils prés\frflex{a}gent le pire :
\item « C’est un mal pernicie\frflex{u}x qui le ronge ;~\psstar{} le voilà couché, il ne pourra pl\frflex{u}s se lever. »
\item Même l’ami, qui av\frflex{a}it ma confiance~\psstar{} et partageait mon pain, m’a frapp\frflex{é} du talon.
\item Mais toi, Seigneur, prends piti\frflex{é} de moi ;~\psstar{} relève-moi, je leur rendr\frflex{a}i ce qu’ils méritent.
\item Oui, je saur\frflex{a}i que tu m’aimes~\psstar{} si mes ennemis ne chantent p\frflex{a}s victoire.
\item Dans mon innocence tu m’\frflex{a}s soutenu~\psstar{} et rétabli pour toujo\frflex{u}rs devant ta face.
\item Béni soit le Seigneur, Die\frflex{u} d’Israël,~\psstar{} depuis toujours et pour toujours ! Am\frflex{e}n ! Amen !