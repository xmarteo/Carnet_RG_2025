\item Tu me scrutes, Seigneur, et tu sais !\pscross{} Tu sais quand je m’ass\frflex{o}is, quand je me lève ;\psstar{} de très loin, tu pén\frflex{è}tres mes pensées.
\item Que je marche ou me rep\frflex{o}se, tu le vois,\psstar{} tous mes chemins te s\frflex{o}nt familiers.
\item Avant qu’un mot ne parvi\frflex{e}nne à mes lèvres,\psstar{} déjà, Seigne\frflex{u}r, tu le sais.
\item Tu me devances et me poursu\frflex{i}s, tu m’enserres,\psstar{} tu as mis la m\frflex{a}in sur moi.
\item Savoir prodigie\frflex{u}x qui me dépasse,\psstar{} hauteur que je ne pu\frflex{i}s atteindre !
\item Où donc aller, l\frflex{o}in de ton souffle ?\psstar{} où m’enfuir, l\frflex{o}in de ta face ?
\item Je gravis les cie\frflex{u}x : tu es là ;\psstar{} je descends chez les m\frflex{o}rts : te voici.
\item Je prends les \frflex{a}iles de l’aurore\psstar{} et me pose au-del\frflex{à} des mers :
\item Même là, ta m\frflex{a}in me conduit,\psstar{} ta main dr\frflex{o}ite me saisit.
\item J’avais dit : « Les tén\frflex{è}bres m’écrasent ! »\psstar{} mais la nuit devient lumi\frflex{è}re autour de moi.
\item Même la ténèbre pour t\frflex{o}i n’est pas ténèbre,\psstar{} et la nuit comme le jo\frflex{u}r est lumière !
\item C’est toi qui as cré\frflex{é} mes reins,\psstar{} qui m’as tissé dans le s\frflex{e}in de ma mère.