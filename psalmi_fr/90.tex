\item Quand je me tiens sous l’abr\frflex{i} du Très-Haut\psstar{} et repose à l’\frflex{o}mbre du Puissant,
\item Je dis au Seigne\frflex{u}r : « Mon refuge,\psstar{} mon rempart, mon Die\frflex{u}, dont je suis sûr ! »
\item C’est lui qui te sauve des filets du chasseur et de la p\frflex{e}ste maléfique ;\psstar{} il te co\frflex{u}vre et te protège.
\item Tu trouves sous son \frflex{a}ile un refuge :\psstar{} sa fidélité est une arm\frflex{u}re, un bouclier.
\item Tu ne craindras ni les terre\frflex{u}rs de la nuit,\psstar{} ni la flèche qui v\frflex{o}le au grand jour,
\item Ni la peste qui r\frflex{ô}de dans le noir,\psstar{} ni le fléau qui fr\frflex{a}ppe à midi.
\item Qu’il en tombe mille à tes côtés,\pscross{} qu’il en tombe dix m\frflex{i}lle à ta droite,\psstar{} toi, tu r\frflex{e}stes hors d’atteinte.
\item Il suffit que tu o\frflex{u}vres les yeux,\psstar{} tu verras le sal\frflex{a}ire du méchant.
\item Oui, le Seigne\frflex{u}r est ton refuge ;\psstar{} tu as fait du Très-Ha\frflex{u}t ta forteresse.
\item Le malheur ne pourr\frflex{a} te toucher,\psstar{} ni le danger, approch\frflex{e}r de ta demeure :
\item Il donne missi\frflex{o}n à ses anges\psstar{} de te garder sur to\frflex{u}s tes chemins.
\item Ils te porter\frflex{o}nt sur leurs mains\psstar{} pour que ton pied ne he\frflex{u}rte les pierres ;
\item Tu marcheras sur la vip\frflex{è}re et le scorpion,\psstar{} tu écraseras le li\frflex{o}n et le Dragon.
\item « Puisqu’il s’attache à m\frflex{o}i, je le délivre ;\psstar{} je le défends, car il conn\frflex{a}ît mon Nom.
\item Il m’appelle, et m\frflex{o}i, je lui réponds ;\psstar{} je suis avec lu\frflex{i} dans son épreuve.
\item Je veux le libérer, le glorifier ;\pscross{} de longs jours, je ve\frflex{u}x le rassasier,\psstar{} et je ferai qu’il v\frflex{o}ie mon salut. »