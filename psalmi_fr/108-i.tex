\item Dieu de ma louange, s\frflex{o}rs de ton silence !\psstar{} La bouche de l'impie, la bouche du fourbe, s'o\frflex{u}vrent contre moi.
\item Ils parlent de moi pour d\frflex{i}re des mensonges ;\psstar{} ils me cernent de propos haineux, ils m'att\frflex{a}quent sans raison.
\item Pour prix de mon amiti\frflex{é}, ils m'accusent,\psstar{} moi qui ne su\frflex{i}s que prière.
\item Ils me rendent le m\frflex{a}l pour le bien,\psstar{} ils paient mon amiti\frflex{é} de leur haine.
\item « Chargeons un imp\frflex{i}e de l'attaquer :\psstar{} qu'un accusateur se ti\frflex{e}nne à sa droite.
\item « À son procès, qu'on l\frflex{e} déclare impie,\psstar{} que sa prière soit compt\frflex{é}e comme une faute.
\item « Que les jours de sa v\frflex{i}e soient écourtés,\psstar{} qu'un autre pr\frflex{e}nne sa charge.
\item « Que ses fils devi\frflex{e}nnent orphelins,\psstar{} que sa f\frflex{e}mme soit veuve.
\item « Qu'ils soient errants, vagab\frflex{o}nds, ses fils,\psstar{} qu'ils mendient, expuls\frflex{é}s de leurs ruines.
\item « Qu'un usurier sais\frflex{i}sse tout son bien,\psstar{} que d'autres s'emparent du fru\frflex{i}t de son travail.
\item « Que nul ne lui r\frflex{e}ste fidèle,\psstar{} que nul n'ait piti\frflex{é} de ses orphelins.
\item « Que soit retranch\frflex{é}e sa descendance, que son nom s'efface av\frflex{e}c ses enfants. »