\item Accuse, Seigne\frflex{u}r, ceux qui m’accusent,\psstar{} attaque ce\frflex{u}x qui m’attaquent.
\item Prends une arm\frflex{u}re, un bouclier,\psstar{} lève-t\frflex{o}i pour me défendre.
\item Brandis la lance et l’épée contre ce\frflex{u}x qui me poursuivent.\psstar{} Parle et dis-moi : « Je su\frflex{i}s ton salut. »
\item Qu’ils soient humiliés, déshonorés, ceux qui s’en pr\frflex{e}nnent à ma vie !\psstar{} Qu’ils reculent, couverts de honte, ceux qui ve\frflex{u}lent mon malheur !
\item Qu’ils soient comme la p\frflex{a}ille dans le vent\psstar{} lorsque l’ange du Seigne\frflex{u}r les balaiera !
\item Que leur chemin soit obsc\frflex{u}r et glissant\psstar{} lorsque l’ange du Seigne\frflex{u}r les chassera !
\item Sans raison ils ont tend\frflex{u} leur filet,\psstar{} et sans raison creusé un tro\frflex{u} pour me perdre.
\item Qu’un désastre imprév\frflex{u} les surprenne,\psstar{} qu’ils soient pris dans le filet qu’ils ont caché, et dans ce dés\frflex{a}stre, qu’ils succombent !
\item Pour moi, le Seigne\frflex{u}r sera ma joie,\psstar{} et son sal\frflex{u}t, mon allégresse !
\item De tout mon \frflex{ê}tre, je dirai :\psstar{} « Qui est comme t\frflex{o}i, Seigneur,
\item Pour arracher un pauvre à plus f\frflex{o}rt que lui,\psstar{} un pauvre, un malheureux, à qu\frflex{i} le dépouille. »