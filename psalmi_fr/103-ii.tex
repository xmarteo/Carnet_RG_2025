\item De tes demeures tu abre\frflex{u}ves les montagnes,\psstar{} et la terre se rassasie du fru\frflex{i}t de tes œuvres ;
\item Tu fais pousser les prair\frflex{i}es pour les troupeaux,\psstar{} et les champs pour l’h\frflex{o}mme qui travaille. 
\item De la terre il t\frflex{i}re son pain :\psstar{} le vin qui réjou\frflex{i}t le cœur de l’homme, 
\item L’huile qui adouc\frflex{i}t son visage,\psstar{} et le pain qui fortif\frflex{i}e le cœur de l’homme.
\item Les arbres du Seigne\frflex{u}r se rassasient,\psstar{} les cèdres qu’il a plant\frflex{é}s au Liban ;
\item C’est là que vient nich\frflex{e}r le passereau,\psstar{} et la cigogne a sa mais\frflex{o}n dans les cyprès ;
\item Aux chamois, les ha\frflex{u}tes montagnes,\psstar{} aux marmottes, l’abr\frflex{i} des rochers.
\item Tu fis la lune qui m\frflex{a}rque les temps\psstar{} et le soleil qui connaît l’he\frflex{u}re de son coucher.
\item Tu fais descendre les tén\frflex{è}bres, la nuit vient :\psstar{} les animaux dans la for\frflex{ê}t s’éveillent ;
\item Le lionceau rug\frflex{i}t vers sa proie,\psstar{} il réclame à Die\frflex{u} sa nourriture.
\item Quand paraît le sol\frflex{e}il, ils se retirent :\psstar{} chacun g\frflex{a}gne son repaire.
\item L’homme s\frflex{o}rt pour son ouvrage,\psstar{} pour son trav\frflex{a}il, jusqu’au soir.