\item Toi, tu montreras ta tendr\frflex{e}sse pour Sion ;~\psstar{} il est temps de la prendre en pitié : l’he\frflex{u}re est venue.
\item Tes serviteurs ont piti\frflex{é} de ses ruines,~\psstar{} ils aiment j\frflex{u}squ’à sa poussière.
\item Les nations craindront le N\frflex{o}m du Seigneur,~\psstar{} et tous les rois de la t\frflex{e}rre, sa gloire :
\item Quand le Seigneur rebâtir\frflex{a} Sion,~\psstar{} quand il apparaîtr\frflex{a} dans sa gloire,
\item Il se tournera vers la pri\frflex{è}re du spolié,~\psstar{} il n’aura pas mépris\frflex{é} sa prière.
\item Que cela soit écrit pour l’\frflex{â}ge à venir,~\psstar{} et le peuple à nouveau créé chanter\frflex{a} son Dieu :
\item « Des hauteurs, son sanctuaire, le Seigne\frflex{u}r s’est penché ;~\psstar{} du ciel, il reg\frflex{a}rde la terre
\item Pour entendre la pl\frflex{a}inte des captifs~\psstar{} et libérer ceux qui dev\frflex{a}ient mourir. »
\item On publiera dans Sion le N\frflex{o}m du Seigneur~\psstar{} et sa louange dans to\frflex{u}t Jérusalem,
\item Au rassemblement des roya\frflex{u}mes et des peuples~\psstar{} qui viendront serv\frflex{i}r le Seigneur.