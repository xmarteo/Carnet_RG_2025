\item Au bord des fleuves de Babylone nous étions assis et nous pleurions,\pscross{} nous souven\frflex{a}nt de Sion ;\psstar{} aux saules des alentours nous avi\frflex{o}ns pendu nos harpes.
\item C’est là que nos vainqueurs nous demandèrent des chansons,\pscross{} et nos bourrea\frflex{u}x, des airs joyeux :\psstar{} « Chantez-nous, disaient-ils, quelque ch\frflex{a}nt de Sion. »
\item Comment chanterions-nous un chant du Seigneur\pscross{} sur une t\frflex{e}rre étrangère ?\psstar{} Si je t’oublie, Jérusalem, que ma main dr\frflex{o}ite m’oublie !
\item Je veux que ma langue s’attache à mon palais\pscross{} si je p\frflex{e}rds ton souvenir,\psstar{} si je n’élève Jérusalem, au somm\frflex{e}t de ma joie.
\item Souviens-toi, Seigneur, des fils du pays d’Édom,\pscross{} et de ce jo\frflex{u}r à Jérusalem\psstar{} où ils criaient : « Détruisez-la, détruisez-l\frflex{a} de fond en comble ! »
\item Ô Babylone misérable,\pscross{} heureux qui te revaudra les ma\frflex{u}x que tu nous valus ;\psstar{} heureux qui saisira tes enfants, pour les bris\frflex{e}r contre le roc !