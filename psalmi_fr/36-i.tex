\item Ne t’indigne pas à la v\frflex{u}e des méchants,\psstar{} n’envie pas les g\frflex{e}ns malhonnêtes ;
\item Aussi vite que l’h\frflex{e}rbe, ils se fanent ;\psstar{} comme la verd\frflex{u}re, ils se flétrissent.
\item Fais confiance au Seigne\frflex{u}r, agis bien,\psstar{} habite la terre et r\frflex{e}ste fidèle ;
\item Mets ta j\frflex{o}ie dans le Seigneur :\psstar{} il comblera les dés\frflex{i}rs de ton cœur.
\item Dirige ton chem\frflex{i}n vers le Seigneur,\psstar{} fais-lui confiance, et lu\frflex{i}, il agira.
\item Il fera lever comme le jo\frflex{u}r ta justice,\psstar{} et ton droit comme le pl\frflex{e}in midi.
\item Repose-t\frflex{o}i sur le Seigneur\psstar{} et c\frflex{o}mpte sur lui. 
\item Ne t’indigne pas devant celu\frflex{i} qui réussit,\psstar{} devant l’homme qui \frflex{u}se d’intrigues.
\item Laisse ta colère, c\frflex{a}lme ta fièvre,\psstar{} ne t’indigne pas : il n’en viendr\frflex{a}it que du mal ;
\item Les méchants ser\frflex{o}nt déracinés,\psstar{} mais qui espère le Seigneur posséder\frflex{a} la terre.
\item Encore un peu de t\frflex{e}mps : plus d’impie ;\psstar{} tu pénètres chez lu\frflex{i} : il n’y est plus.
\item Les doux posséder\frflex{o}nt la terre\psstar{} et jouiront d’une abond\frflex{a}nte paix.
\item L’impie peut intrigu\frflex{e}r contre le juste\psstar{} et grincer des d\frflex{e}nts contre lui,
\item Le Seigneur se m\frflex{o}que du méchant\psstar{} car il voit son jo\frflex{u}r qui arrive.
\item L’impie a tiré son épée, il a tend\frflex{u} son arc\psstar{} pour abattre le pauvre et le faible, pour tu\frflex{e}r l’homme droit.
\item Mais l’épée lui entrer\frflex{a} dans le cœur,\psstar{} et son \frflex{a}rc se brisera.