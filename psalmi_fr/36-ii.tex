\item Pour le juste, avoir pe\frflex{u} de biens~\psstar{} vaut mieux que la fort\frflex{u}ne des impies.
\item Car le bras de l’imp\frflex{i}e sera brisé,~\psstar{} mais le Seigneur souti\frflex{e}nt les justes.
\item Il connaît les jo\frflex{u}rs de l’homme intègre~\psstar{} qui recevra un hérit\frflex{a}ge impérissable.
\item Pas de honte pour lu\frflex{i} aux mauvais jours ;~\psstar{} aux temps de famine, il ser\frflex{a} rassasié.
\item Mais oui, les imp\frflex{i}es disparaîtront~\psstar{} comme la par\frflex{u}re des prés ; 
\item C’en est fini des ennem\frflex{i}s du Seigneur :~\psstar{} ils s’en v\frflex{o}nt en fumée.
\item L’impie empr\frflex{u}nte et ne rend pas ;~\psstar{} le juste a piti\frflex{é} : il donne.
\item Ceux qu’il bénit posséder\frflex{o}nt la terre,~\psstar{} ceux qu’il maudit ser\frflex{o}nt déracinés.
\item Quand le Seigneur condu\frflex{i}t les pas de l’homme,~\psstar{} ils sont fermes et sa m\frflex{a}rche lui plaît.
\item S’il trébuche, il ne t\frflex{o}mbe pas~\psstar{} car le Seigneur le souti\frflex{e}nt de sa main.
\item Jamais, de ma jeun\frflex{e}sse à mes vieux jours,~\psstar{} je n’ai vu le juste abandonné ni ses enfants mendi\frflex{e}r leur pain.
\item Chaque jour il a piti\frflex{é}, il prête ;~\psstar{} ses descend\frflex{a}nts seront bénis.
\item Évite le mal, f\frflex{a}is ce qui est bien,~\psstar{} et tu auras une habitati\frflex{o}n pour toujours,
\item Car le Seigneur \frflex{a}ime le bon droit,~\psstar{} il n’abandonne p\frflex{a}s ses amis. 
\item Ceux-là seront préserv\frflex{é}s à jamais,~\psstar{} les descendants de l’impie ser\frflex{o}nt déracinés.
\item Les justes posséder\frflex{o}nt la terre~\psstar{} et toujo\frflex{u}rs l’habiteront.