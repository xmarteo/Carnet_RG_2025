\item Lève-toi, Seigneur ! Die\frflex{u}, étends la main !\psstar{} N’oublie p\frflex{a}s le pauvre !
\item Pourquoi l’impie brave-t-\frflex{i}l le Seigneur\psstar{} en lui disant : « Viendras-t\frflex{u} me chercher ? »
\item Mais tu as vu : tu regardes le m\frflex{a}l et la souffrance,\psstar{} tu les pr\frflex{e}nds dans ta main ;
\item Sur toi rep\frflex{o}se le faible,\psstar{} c’est toi qui viens en \frflex{a}ide à l’orphelin.
\item Brise le bras de l’imp\frflex{i}e, du méchant ;\psstar{} alors tu chercheras son impiét\frflex{é} sans la trouver.
\item À tout jamais, le Seigne\frflex{u}r est roi :\psstar{} les païens ont pér\frflex{i} sur sa terre.
\item Tu entends, Seigneur, le dés\frflex{i}r des pauvres,\psstar{} tu rassures leur cœ\frflex{u}r, tu les écoutes.
\item Que justice soit rendue à l’orphelin, qu’il n’y ait pl\frflex{u}s d’opprimé,\psstar{} et que tremble le mortel, n\frflex{é} de la terre !