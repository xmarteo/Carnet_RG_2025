\item Qu’il est bon de rendre gr\frflex{â}ce au Seigneur,~\psstar{} de chanter pour ton N\frflex{o}m, Dieu Très-Haut,
\item D’annoncer dès le mat\frflex{i}n ton amour,~\psstar{} ta fidélit\frflex{é}, au long des nuits,
\item Sur la lyre à dix c\frflex{o}rdes et sur la harpe,~\psstar{} sur un murm\frflex{u}re de cithare.
\item Tes œuvres me c\frflex{o}mblent de joie ;~\psstar{} devant l’ouvrage de tes m\frflex{a}ins, je m’écrie :
\item « Que tes œuvres sont gr\frflex{a}ndes, Seigneur !~\psstar{} Combien sont prof\frflex{o}ndes tes pensées ! »
\item L’homme born\frflex{é} ne le sait pas,~\psstar{} l’insensé ne pe\frflex{u}t le comprendre :
\item Les impies cr\frflex{o}issent comme l’herbe,~\psstar{} ils fleurissent, ceux qui font le mal, mais pour dispar\frflex{a}ître à tout jamais.
\item Toi, qui hab\frflex{i}tes là-haut,~\psstar{} tu es pour toujo\frflex{u}rs le Seigneur.
\item Vois tes ennemis, Seigneur, vois tes ennem\frflex{i}s qui périssent,~\psstar{} et la déroute de ce\frflex{u}x qui font le mal.
\item Tu me donnes la fo\frflex{u}gue du taureau,~\psstar{} tu me baignes d’hu\frflex{i}le nouvelle ;
\item J’ai vu, j’ai repér\frflex{é} mes espions,~\psstar{} j’entends ceux qui vi\frflex{e}nnent m’attaquer.
\item Le juste grandir\frflex{a} comme un palmier,~\psstar{} il poussera comme un c\frflex{è}dre du Liban ;
\item Planté dans les parv\frflex{i}s du Seigneur,~\psstar{} il grandira dans la mais\frflex{o}n de notre Dieu.
\item Vieillissant, il fructif\frflex{i}e encore,~\psstar{} il garde sa s\frflex{è}ve et sa verdeur
\item Pour annoncer : « Le Seigne\frflex{u}r est droit !~\psstar{} Pas de ruse en Die\frflex{u}, mon rocher ! »