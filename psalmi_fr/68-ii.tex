\item Dans ton grand amour, Die\frflex{u}, réponds-moi,\psstar{} par ta vérit\frflex{é} sauve-moi.
\item Tire-moi de la boue, sin\frflex{o}n je m’enfonce :\psstar{} que j’échappe à ceux qui me haïssent, à l’ab\frflex{î}me des eaux.
\item Que les flots ne me submergent pas, que le go\frflex{u}ffre ne m’avale,\psstar{} que la gueule du puits ne se ferme p\frflex{a}s sur moi.
\item Réponds-moi, Seigneur, car il est b\frflex{o}n, ton amour ;\psstar{} dans ta grande tendresse, r\frflex{e}garde-moi.
\item Ne cache pas ton vis\frflex{a}ge à ton serviteur ;\psstar{} je suffoque : v\frflex{i}te, réponds-moi. 
\item Sois proche de m\frflex{o}i, rachète-moi,\psstar{} paie ma ranç\frflex{o}n à l’ennemi.
\item Toi, tu le sais, on m’insulte : je suis bafou\frflex{é}, déshonoré ;\psstar{} tous mes oppresseurs sont l\frflex{à}, devant toi.
\item L’insulte m’a broy\frflex{é} le cœur,\psstar{} le m\frflex{a}l est incurable ;
\item J’espérais un seco\frflex{u}rs, mais en vain,\psstar{} des consolateurs, je n’en ai p\frflex{a}s trouvé.
\item À mon pain, ils ont mêl\frflex{é} du poison ;\psstar{} quand j’avais soif, ils m’ont donn\frflex{é} du vinaigre.
\item Que leur table devi\frflex{e}nne un piège,\psstar{} un guet-ap\frflex{e}ns pour leurs convives !
\item Que leurs yeux aveugl\frflex{é}s ne voient plus,\psstar{} qu’à tout instant les r\frflex{e}ins leur manquent !
\item Déverse sur e\frflex{u}x ta fureur,\psstar{} que le feu de ta col\frflex{è}re les saisisse,
\item Que leur camp devi\frflex{e}nne un désert,\psstar{} que nul n’hab\frflex{i}te sous leurs tentes !
\item Celui que tu frapp\frflex{a}is, ils le pourchassent\psstar{} en comptant les co\frflex{u}ps qu’il reçoit.
\item Charge-les, fa\frflex{u}te sur faute ;\psstar{} qu’ils n’aient pas d’acc\frflex{è}s à ta justice.
\item Qu’ils soient rayés du l\frflex{i}vre de vie,\psstar{} retranchés du n\frflex{o}mbre des justes.