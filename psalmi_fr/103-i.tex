\item Bénis le Seigne\frflex{u}r, ô mon âme ;\psstar{} Seigneur mon Die\frflex{u}, tu es si grand ! 
\item Revêt\frflex{u} de magnificence,\psstar{} tu as pour mantea\frflex{u} la lumière ! 
\item Comme une tenture, tu dépl\frflex{o}ies les cieux,\psstar{} tu élèves dans leurs ea\frflex{u}x tes demeures ; 
\item Des nuées, tu te f\frflex{a}is un char,\psstar{} tu t’avances sur les \frflex{a}iles du vent ;
\item Tu prends les v\frflex{e}nts pour messagers,\psstar{} pour serviteurs, les fl\frflex{a}mmes des éclairs.
\item Tu as donné son ass\frflex{i}se à la terre :\psstar{} qu’elle reste inébranl\frflex{a}ble au cours des temps.
\item Tu l’as vêtue de l’ab\frflex{î}me des mers :\psstar{} les eaux couvraient m\frflex{ê}me les montagnes ;
\item À ta menace, elles pr\frflex{e}nnent la fuite,\psstar{} effrayées par le tonn\frflex{e}rre de ta voix.
\item Elles passent les montagnes, se r\frflex{u}ent dans les vallées\psstar{} vers le lieu que tu leur \frflex{a}s préparé.
\item Tu leur imposes la lim\frflex{i}te à ne pas franchir :\psstar{} qu’elles ne reviennent jam\frflex{a}is couvrir la terre.
\item Dans les ravins tu fais jaill\frflex{i}r des sources\psstar{} et l’eau chemine au cre\frflex{u}x des montagnes ;
\item Elle abreuve les b\frflex{ê}tes des champs :\psstar{} l’âne sauvage y c\frflex{a}lme sa soif ;
\item Les oiseaux séjo\frflex{u}rnent près d’elle :\psstar{} dans le feuillage on ent\frflex{e}nd leurs cris.