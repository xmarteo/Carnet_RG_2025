\item J’ai dit : « Je garder\frflex{a}i mon chemin~\psstar{} sans laisser ma l\frflex{a}ngue s’égarer ; 
\item Je garderai un bâill\frflex{o}n sur ma bouche,~\psstar{} tant que l’impie se tiendr\frflex{a} devant moi. »
\item Je suis resté muet, silencieux ; je me tais\frflex{a}is, mais sans profit.~\psstar{} Mon tourment s’exaspérait, mon cœur brûl\frflex{a}it en moi. 
\item Quand j’y pens\frflex{a}is, je m’enflammais,~\psstar{} et j’ai laissé parl\frflex{e}r ma langue.
\item Seigneur, fais-moi connaître ma fin, quel est le n\frflex{o}mbre de mes jours :~\psstar{} je connaîtrai combi\frflex{e}n je suis fragile.
\item Vois le peu de jo\frflex{u}rs que tu m’accordes :~\psstar{} ma durée n’est ri\frflex{e}n devant toi. 
\item L’homme ici-b\frflex{a}s n’est qu’un souffle ;~\psstar{} il va, il vient, il n’\frflex{e}st qu’une image. 
\item Rien qu’un souffle, to\frflex{u}s ses tracas ;~\psstar{} il amasse, mais qu\frflex{i} recueillera ?
\item Maintenant, que puis-je att\frflex{e}ndre, Seigneur ?~\psstar{} Elle est en t\frflex{o}i, mon espérance.
\item Délivre-moi de to\frflex{u}s mes péchés,~\psstar{} épargne-moi les inj\frflex{u}res des fous.
\item Je me suis tu, je n’ouvre p\frflex{a}s la bouche,~\psstar{} car c’est t\frflex{o}i qui es à l’œuvre.
\item Éloigne de m\frflex{o}i tes coups :~\psstar{} je succombe sous ta m\frflex{a}in qui me frappe.
\item Tu redresses l’homme en corrigeant sa faute,\pscross{} tu ronges comme un v\frflex{e}r son désir ;~\psstar{} l’h\frflex{o}mme n’est qu’un souffle.
\item Entends ma prière, Seigneur, éco\frflex{u}te mon cri ;~\psstar{} ne reste pas so\frflex{u}rd à mes pleurs. 
\item Je ne suis qu’un h\frflex{ô}te chez toi,~\psstar{} un passant, comme to\frflex{u}s mes pères.
\item Détourne de moi tes ye\frflex{u}x, que je respire~\psstar{} avant que je m’en \frflex{a}ille et ne sois plus.