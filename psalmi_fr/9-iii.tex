\item Pourquoi, Seigne\frflex{u}r, es-tu si loin ?\psstar{} Pourquoi te cach\frflex{e}r aux jours d’angoisse ?
\item L’impie, dans son orgueil, poursu\frflex{i}t les malheureux :\psstar{} ils se font prendre aux r\frflex{u}ses qu’il invente.
\item L’impie se glorifie du dés\frflex{i}r de son âme,\psstar{} l’arrogant blasphème, il br\frflex{a}ve le Seigneur ;
\item Plein de suffisance, l’imp\frflex{i}e ne cherche plus :\psstar{} « Dieu n’est rien », voil\frflex{à} toute sa ruse.
\item À tout moment, ce qu’il fait réussit ;\pscross{} tes sentences le dom\frflex{i}nent de très haut.\psstar{} Tous ses advers\frflex{a}ires, il les méprise.
\item Il s’est dit : « Rien ne pe\frflex{u}t m’ébranler,\psstar{} je suis pour longtemps à l’abr\frflex{i} du malheur. »
\item Sa bouche qui maudit n’est que fra\frflex{u}de et violence,\psstar{} sa langue, mens\frflex{o}nge et blessure.
\item Il se tient à l’aff\frflex{û}t près des villages,\psstar{} il se cache pour tu\frflex{e}r l’innocent.
\item Des yeux, il ép\frflex{i}e le faible,\psstar{} il se cache à l’affût, comme un li\frflex{o}n dans son fourré ;
\item Il se tient à l’affût pour surpr\frflex{e}ndre le pauvre,\psstar{} il attire le pauvre, il le pr\frflex{e}nd dans son filet.
\item Il se b\frflex{a}isse, il se tapit ;\psstar{} de tout son poids, il t\frflex{o}mbe sur le faible.
\item Il dit en lui-même : « Die\frflex{u} oublie !\psstar{} il couvre sa face, jam\frflex{a}is il ne verra ! »