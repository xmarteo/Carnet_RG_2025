\item Qu’ils sont gr\frflex{a}nds, tes bienfaits !\psstar{} Tu les réserves à ce\frflex{u}x qui te craignent.
\item Tu combles, à la f\frflex{a}ce du monde,\psstar{} ceux qui ont en t\frflex{o}i leur refuge.
\item Tu les caches au plus secr\frflex{e}t de ta face,\psstar{} loin des intr\frflex{i}gues des hommes.
\item Tu leur rés\frflex{e}rves un lieu sûr,\psstar{} loin des l\frflex{a}ngues méchantes.
\item Béni soit le Seigneur :\pscross{} son amour a fait pour m\frflex{o}i des merveilles\psstar{} dans la v\frflex{i}lle retranchée !
\item Et moi, dans mon tro\frflex{u}ble, je disais :\psstar{} « Je ne suis pl\frflex{u}s devant tes yeux. »
\item Pourtant, tu écout\frflex{a}is ma prière\psstar{} quand je cri\frflex{a}is vers toi.
\item Aimez le Seigneur, vous, ses fidèles :\pscross{} le Seigneur v\frflex{e}ille sur les siens ;\psstar{} mais il rétribue avec rigueur qui se m\frflex{o}ntre arrogant.
\item Soyez f\frflex{o}rts, prenez courage,\psstar{} vous tous qui espér\frflex{e}z le Seigneur !