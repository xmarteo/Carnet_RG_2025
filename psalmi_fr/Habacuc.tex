\item Seigneur, j’ai entend\frflex{u} parler de toi ;\psstar{} devant ton œuvre, Seigne\frflex{u}r, j’ai craint !
\item Dans le cours des ann\frflex{é}es, fais-la revivre,\psstar{} dans le cours des ann\frflex{é}es, fais-la connaître !
\item Quand tu frém\frflex{i}s de colère,\psstar{} tu te souvi\frflex{e}ns d’avoir pitié.
\item Dieu vi\frflex{e}nt de Témane\psstar{} et le saint, du m\frflex{o}nt de Parane ;
\item Sa majesté co\frflex{u}vre les cieux,\psstar{} sa louange empl\frflex{i}t la terre.
\item Son éclat est pareil à la lumière ;\pscross{} deux rayons s\frflex{o}rtent de ses mains :\psstar{} là se tient cach\frflex{é}e sa puissance.
\item Devant lui m\frflex{a}rche la peste,\psstar{} et la fièvre met ses p\frflex{a}s dans les siens.
\item Il s’arrête, et la t\frflex{e}rre tremble,\psstar{} il regarde et fait sursaut\frflex{e}r les nations.
\item Les montagnes de toujours se disloquent,\pscross{} les collines d’autref\frflex{o}is s’effondrent,\psstar{} qui furent autrefois des r\frflex{o}utes pour lui.
\item J’ai vu les tentes de Koush\frflex{a}ne dans la misère ;\psstar{} les abris du pays de Madi\frflex{a}ne chancellent !
\item Est-ce contre les fleuves, Seigneur, que fl\frflex{a}mbe ta colère,\psstar{} contre les fleuves, contre la m\frflex{e}r, ta fureur,
\item Pour que tu m\frflex{o}ntes sur tes chevaux,\psstar{} sur tes ch\frflex{a}rs de victoire ?
\item Tu sors ton arc, tu le ti\frflex{e}ns en éveil,\psstar{} tu le rassasies des tra\frflex{i}ts de ta parole. 
\item Par des fleuves, tu rav\frflex{i}nes la terre.\psstar{} Les montagnes t’ont v\frflex{u} : elles tremblent. 
\item Une trombe d’e\frflex{a}u a passé, l’Abîme a donn\frflex{é} de la voix. 
\item Le soleil, là-haut, a élev\frflex{é} ses mains,\psstar{} la lune s’est arrêt\frflex{é}e en sa demeure, 
\item À la lueur de tes fl\frflex{è}ches qui volent,\psstar{} à la clarté des écla\frflex{i}rs de ta lance.
\item Dans ton indignation, tu parco\frflex{u}rs la terre ;\psstar{} dans ta colère, tu piét\frflex{i}nes des nations.
\item Tu es sorti pour sauv\frflex{e}r ton peuple,\psstar{} pour sauv\frflex{e}r ton messie.
\item Tu as décapité la mais\frflex{o}n du méchant,\psstar{} tu l’as dénud\frflex{é}e de fond en comble.
\item Tu as percé de ses traits le ch\frflex{e}f de ses guerriers ;\psstar{} ils se déchaînaient pour me disperser, joyeusement, comme pour dévorer dans leur repa\frflex{i}re un malheureux.
\item Tu as foulé, de tes cheva\frflex{u}x, la mer\psstar{} et le remo\frflex{u}s des eaux profondes.
\item J’ai entendu et mes entrailles ont frémi ;\pscross{} à cette voix, mes l\frflex{è}vres tremblent,\psstar{} la carie pén\frflex{è}tre mes os.
\item Et moi je frémis d’être là,\pscross{} d’attendre en silence le jo\frflex{u}r d’angoisse\psstar{} qui se lèvera sur le peuple dress\frflex{é} contre nous.
\item Le figui\frflex{e}r n’a pas fleuri ;\psstar{} pas de réc\frflex{o}lte dans les vignes.
\item Le fruit de l’olivi\frflex{e}r a déçu ;\psstar{} dans les champs, pl\frflex{u}s de nourriture.
\item L’enclos s’est vid\frflex{é} de ses brebis,\psstar{} et l’ét\frflex{a}ble, de son bétail.
\item Et moi, je bondis de j\frflex{o}ie dans le Seigneur,\psstar{} j’exulte en Die\frflex{u}, mon Sauveur !
\item Le Seigneur mon Dieu est ma force ;\pscross{} il me donne l’agilit\frflex{é} du chamois,\psstar{} il me fait march\frflex{e}r dans les hauteurs.