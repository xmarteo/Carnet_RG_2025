\item Pour toi, je chanter\frflex{a}i un chant nouveau,~\psstar{} pour toi, je jouerai sur la h\frflex{a}rpe à dix cordes,
\item Pour toi qui donnes aux r\frflex{o}is la victoire~\psstar{} et sauves de l’épée meurtrière Dav\frflex{i}d, ton serviteur.
\item Délivre-m\frflex{o}i, sauve-moi~\psstar{} de l’emprise d’un pe\frflex{u}ple étranger : 
\item Il dit des par\frflex{o}les mensongères,~\psstar{} sa main est une m\frflex{a}in parjure.
\item Que nos fils soient par\frflex{e}ils à des plants~\psstar{} bien ven\frflex{u}s dès leur jeune âge ;
\item Nos filles, par\frflex{e}illes à des colonnes~\psstar{} sculpt\frflex{é}es pour un palais !
\item Nos greniers, rempl\frflex{i}s, débordants,~\psstar{} regorger\frflex{o}nt de biens ; 
\item Les troupeaux, par milli\frflex{e}rs, par myriades,~\psstar{} emplir\frflex{o}nt nos campagnes !
\item Nos vassaux nous resteront soumis, pl\frflex{u}s de défaites ;~\psstar{} plus de brèches dans nos murs, plus d’alert\frflex{e}s sur nos places !
\item Heureux le pe\frflex{u}ple ainsi comblé !~\psstar{} Heureux le peuple qui a pour Die\frflex{u} « Le Seigneur » !