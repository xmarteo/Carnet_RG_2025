% !TEX TS-program = lualatex
% !TEX encoding = UTF-8

\documentclass[Carnet_RG_2025.tex]{subfiles}

\ifcsname preamble@file\endcsname
  \setcounter{page}{\getpagerefnumber{M-20251129_01_Matines}}
\fi

\begin{document}
\bigtitle{Matines de la Sainte Vierge le samedi}{Samedi 29 novembre}{Matines}

\smallscore{or_domine_labia_ferialis}{\vvrub Seigneur, ouvre mes lèvres \rrrub Et ma bouche annoncera ta louange. \vvrub Dieu, viens à mon aide, \rrrub Seigneur, viens vite à mon secours. Gloire au Père, et au Fils, et au Saint-Esprit, comme il était au commencement, maintenant et toujours, dans les siècles des siècles. Amen.}

\gscore{inv_ave_maria}{\aarub Je vous salue Marie, pleine de grâce, le Seigneur est avec vous.}

\rubric{On répète l'invitatoire, puis, après chaque verset du psaume, on chante l'invitatoire entier ou sa deuxième partie.}

\gscore{hy_quem_terra_pontus}{}
\translation{
Celui que terre, mer, astres
vénèrent, adorent, annoncent,
Celui qui régit ce triple monde,
Marie le porte caché dans son sein.

Celui que lune, soleil et toutes choses
servent en tout temps,
est porté par les entrailles d’une jeune vierge,
toute pénétrée de la grâce céleste.

La bienheureuse mère, par la grâce,
dans l’arche de son sein,
renferme l’Artisan suprême
qui tient le monde dans sa main

Bienheureuse, à la parole d’un messager du ciel,
féconde par le Saint-Esprit,
et son sein donne au monde
le désiré des nations.

O Jésus, gloire à vous
qui êtes né de la Vierge,
ainsi qu’au Père et à l’Esprit nourricier,
dans les siècles éternels.
}

\gscore{an_memor_fuit}{Il s'est toujours souvenu de son alliance, le Seigneur notre Dieu.}
\psalmVulgate{104-i}
\gscore{an_auxit_dominus}{Le Seigneur fait croître son peuple, et le rend plus fort que ses ennemis.}
\psalmVulgate{104-ii}
\gscore{an_eduxit_deus}{Dieu fit sortir son peuple dans l'exultation, et les élus dans la joie.}
\psalmVulgate{104-iii}
\gscore{an_salvavit_eos}{Le Seigneur les a sauvés à cause de son nom.}
\psalmVulgate{105-i}
\gscore{an_obliti_sunt}{Ils ont oublié Dieu qui les a sauvés.}
\psalmVulgate{105-ii}
\gscore{an_cum_tribularentur}{Quand ils étaient dans l'épreuve, le Seigneur le vit, et il exauça leur prière.}
\psalmVulgate{105-iii}
\gscore{an_clamaverunt}{Ils ont crié vers le Seigneur, et il les a délivrés de leurs angoisses.}
\psalmVulgate{106-i}
\gscore{an_ipsi_viderunt}{Ils ont vu de leurs yeux les oeuvres de Dieu et ses merveilles.}
\psalmVulgate{106-ii}
\gscore{an_videbunt_recti}{Les hommes droits verront et se réjouiront, et ils comprendront les miséricordes du Seigneur.}
\psalmVulgate{106-iii}

\versiculus{Exáltent Dóminum in ecclésia plebis.}{Et in cáthedra seniórum laudent eum.}{Qu'on exalte le Seigneur dans l'assemblée du peuple.}{Et que dans leur chaires, les anciens le louent.}

\patersecreto

\absolutio{Précibus et méritis beátæ Maríæ semper Vírginis et ómni\sylprep{um} \sylprep{San}\sylac{ctó}rum,~\psstar{} perdúcat nos Dóminus ad regna cæ\sylac{ló}rum.}{Que par les prières et les mérites de la bienheureuse Marie toujours Vierge et de tous les saints, le Seigneur nous conduise au royaume des Cieux.}

\benedictio{Nos cum prole pia benedícat Virgo María.}{Que la Vierge Marie nous obtienne la bénédiction de son divin Fils.}

\smalltitle{Première leçon}

\twocoltext{Incipit Malachías Prophéta\\
	\rubric{Ml. 1:1-4}
	Onus verbi Dómini ad Israël in manu Malachíæ. Diléxi vos, dicit Dóminus, et dixístis: In quo dilexísti nos? Nonne frater erat Esau Iacob? dicit Dóminus, et diléxi Iacob, Esau autem ódio hábui, et pósui montes eius in solitúdinem, et hereditátem eius in dracónes desérti? Quod si díxerit Idumǽa: Destrúcti sumus, sed reverténtes ædificábimus quæ destrúcta sunt: hæc dicit Dóminus exercítuum: Isti ædificábunt, et ego déstruam; et vocabúntur términi impietátis, et pópulus cui irátus est Dóminus usque in ætérnum.
	}{
	Commencement du livre de Malachie\\
	~\\
	Parole du Seigneur à Israël par l’intermédiaire de Malachie. Je vous ai aimés, dit le Seigneur, et vous dites : « En quoi nous as-tu aimés ? » Ésaü n’était-il pas frère de Jacob ? – oracle du Seigneur. J’ai eu de l’amour pour Jacob mais je n’ai pas aimé Ésaü. J’ai livré ses montagnes à la désolation, son héritage aux chacals du désert. Si Édom déclare : « Nous avons été détruits, mais nous recommencerons, nous relèverons les ruines », ainsi parle le Seigneur de l’univers : « Qu’ils relèvent, eux ! Moi, je démolirai ! On les appellera “Territoire-de-méchanceté”, “Peuple-qui-met-en-colère-le-Seigneur-pour-toujours”.
	}
	
\tuautem

\gscore{re_laudabilis_populus}{\rrrub Béni soit le peuple que le Seigneur des armées a béni, en disant: «Israël, tu es l'ouvrage de Mes mains, tu es Mon propre héritage.» \vvrub Heureuse la nation dont le Dieu est le Seigneur, et le peuple qu'il a choisi pour son propre héritage.}

\benedictio{Ipsa Virgo Vírginum intercédat pro nobis ad Dóminum.}{Que la Vierge des vierges elle-même intercède pour nous auprès du Seigneur.}

\smalltitle{Deuxième leçon}

\twocoltext{\rubric{Ml. 1: 5-11}\\
	Et óculi vestri vidébunt, et vos dicétis: Magnificétur Dóminus super términum Israël. Fílius honórat patrem, et servus dóminum suum; si ergo Pater ego sum, ubi est honor meus? et, si Dóminus ego sum, ubi est timor meus? dicit Dóminus exercítuum, ad vos, o sacerdótes, qui despícitis nomen meum, et dixístis: In quo despéximus nomen tuum? Offértis super altáre meum panem pollútum, et dícitis: In quo pollúimus te? In eo quod dícitis: Mensa Dómini despécta est. Si offerátis cæcum ad immolándum, nonne malum est? et, si offerátis claudum et lánguidum, nonne malum est? Offer illud duci tuo, si placúerit ei, aut si suscéperit fáciem tuam, dicit Dóminus exercítuum. Et nunc deprecámini vultum Dei, ut misereátur vestri (de manu enim vestra factum est hoc), si quómodo suscípiat fácies vestras, dicit Dóminus exercítuum. Quis est in vobis qui claudat óstia, et incéndat altáre meum gratuíto? Non est mihi volúntas in vobis, dicit Dóminus exercítuum, et munus non suscípiam de manu vestra; Ab ortu enim solis usque ad occásum, magnum est nomen meum in géntibus, et in omni loco sacrificátur: et offértur nómini meo oblátio munda, quia magnum est nomen meum in géntibus, dicit Dóminus exercítuum.
}{Vos yeux le verront et vous direz : “Le Seigneur est grand par-delà le territoire d’Israël !” » Un fils honore son père, et un serviteur, son maître. Si donc je suis père, où est l’honneur qui m’est dû ? Et si je suis maître, où est le respect qui m’est dû ? – déclare le Seigneur de l’univers à vous, les prêtres qui méprisez mon nom. Et vous dites : « En quoi avons-nous méprisé ton nom ? » – En présentant sur mon autel un aliment impur. Mais vous dites : « En quoi t’avons-nous rendu impur ? » – En affirmant : « La table du Seigneur est méprisable ! » Et quand vous présentez au sacrifice une bête aveugle, n’est-ce pas faire le mal ? Et quand vous présentez une bête boiteuse ou malade, n’est-ce pas faire le mal ? Offre-la donc à ton gouverneur ! Sera-t-il content de toi ? Te sera-t-il favorable ? – Le Seigneur de l’univers a parlé. Et maintenant, apaisez donc le visage de Dieu, pour qu’il nous fasse grâce ! Cela est venu de vos mains. Vous sera-t-il favorable ? – Le Seigneur de l’univers a parlé. Qui donc d’entre vous fermera les portes du sanctuaire, pour que vous n’allumiez plus en vain le feu sur mon autel ? Je ne prends aucun plaisir en vous, – dit le Seigneur de l’univers –, je ne désire plus l’offrande de vos mains. Car du levant au couchant du soleil, mon nom est grand parmi les nations. En tout lieu, on brûle de l’encens pour mon nom et on présente une offrande pure, car mon nom est grand parmi les nations, – dit le Seigneur de l’univers.
}

\tuautem

\gscore{re_misit_dominus}{\rrrub Le Seigneur a envoyé son Ange qui a fermé la gueule des lions. Et ils ne m’ont pas fait de mal parce que, devant lui, il ne s’est pas trouvé d’injustice en moi. \vvrub Dieu a envoyé sa miséricorde et sa vérité ; il a arraché mon âme du milieu des lionceaux.}

\benedictio{Per Vírginem matrem concédat nobis Dóminus salútem et pacem.}{Par la Vierge Mère, que le Seigneur nous donne le salut et la paix.}

\smalltitle{Troisième leçon}

\twocoltext{De Expositióne Sancti Basilíi Epíscopi in Isaíam Prophétam\\
\rubric{Cap. 8 post initium}\\
Accéssi, inquit, ad prophetíssam, et in útero accépit et péperit fílium. Quod María prophetíssa fúerit, ad quam próxime accéssit Isaías per prænotiónem spíritus, nemo contradíxerit, qui sit memor verbórum Maríæ, quæ prophético affláta spíritu elocúta est. Quid enim ait? Magníficat ánima mea Dóminum: et exsultávit spíritus meus in Deo, salutári meo. Quia respéxit humilitátem ancíllæ suæ: ecce enim ex hoc beátam me dicent omnes generatiónes. Quod si ánimum accommodáveris univérsis eius verbis, non útique per dissídium negáveris eam fuísse prophetíssam, quod Dómini Spíritus in eam supervénerit, et virtus Altíssimi obumbráverit ei.
}{
Je m'approchai, dit-il, de la Prophétesse, et elle conçut et enfanta un fils. Que Marie soit cette Prophétesse dont approcha Isaïe par la prescience de l’esprit, nul n’y contredira, qui se souvient des paroles que Marie prononça sous l’inspiration de l’esprit prophétique. Que dit-elle en effet ? Mon âme magnifie le Seigneur et mon esprit a exulté en Dieu mon sauveur, parce qu’il a regardé la bassesse de sa servante, et c’est pourquoi toutes les générations me diront bienheureuse. Que si tu appliques ton esprit à toutes ses paroles, certes tu ne nieras point, par esprit de discorde, qu’elle n’ait été Prophétesse, que l’Esprit du Seigneur ne soit survenu en elle et que la puissance du Très-Haut ne l’ait couverte de son ombre.
}

\gscore{or_te_deum}{}

\dominusvobiscum

\oratio{Concéde nos fámulos tuos, quǽsumus, Dómine Deus, perpétua mentis et córporis sanitáte gaudére: et, gloriósa beátæ Maríæ semper Vírginis intercessióne, a præsénti liberári tristítia, et ætérna pérfrui lætítia. Per Dóminum nostrum Iesum Christum, Fílium tuum: qui tecum vivit et regnat in unitáte Spíritus Sancti, Deus, per ómnia sǽcula sæculórum.}{Accorde à tes fidèles, Seigneur Dieu, la perpétuelle santé de l'âme et du corps; et, à l'intercession glorieuse de la bienheureuse Vierge Marie, qu'ils soient libérés des tristesses présentes, et qu'ils goûtent les joies éternelles. Par Jésus Christ, ton Fils, notre Seigneur, qui vit et règne avec toi dans l'unité du Saint-Esprit, Dieu, pour les siècles des siècles.}

\dominusvobiscum

\smallscore{or_benedicamus_FE_BVM_sabbato}{\vvrub Bénissons le Seigneur. \rrrub Nous rendons grâces à Dieu.}

\fideliumanimae

\end{document}