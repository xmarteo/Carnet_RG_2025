% !TEX TS-program = lualatex
% !TEX encoding = UTF-8

\documentclass[Carnet_RG_2025.tex]{subfiles}

\ifcsname preamble@file\endcsname
  \setcounter{page}{\getpagerefnumber{M-20251130_01_Matines}}
\fi

\begin{document}
\bigtitle{Matines du premier dimanche de l'Avent}{Dimanche 30 novembre}{Matines}

\vspace{\baselineskip}

\smallscore{or_domine_labia_festivus}{\vvrub Seigneur, ouvre mes lèvres \rrrub Et ma bouche annoncera ta louange. \vvrub Dieu, viens à mon aide, \rrrub Seigneur, viens vite à mon secours. Gloire au Père, et au Fils, et au Saint-Esprit, comme il était au commencement, maintenant et toujours, dans les siècles des siècles. Amen.}

\smalltitle{Invitatoire}

\gscore{inv_regem_venturum}{\aarub Le Roi qui doit venir, le Seigneur, venez, adorons-le.}

\rubric{On répète l'invitatoire, puis, après chaque verset du psaume, on chante l'invitatoire entier ou sa deuxième partie.}

\twocolitemizedtext{psalmi_vulgata/94_VLrepet.tex}{psalmi_fr/94.tex}

\gscore{hy_verbum_supernum}{}
\hymntranslation{\colored{Ô} Verbe très-haut, tu parais,
Lumière, tu jaillis du Père ;
Tu nais pour secourir le monde
Quand le temps décline en sa course.\\
\colored{E}claire maintenant les coeurs,
Conseume-les de ton amour :
Qu'à l'annonce de ta venue,
Les péchés soient enfin bannis.\\
\colored{E}t lorsque tu viendras en juge
Sonder les actions et les coeurs,
Peser ce qui était caché,
Donner aux justes le Royaume,\\
\colored{P}uissions-nous échapper aux peines
Que notre crime a méritées :
Fais-nous avec les bienheureux
Partager ton ciel pour toujours.
\colored{Ô} Christ, ô Roi plein de bonté,
Gloire à toi et gloire à ton Père,
Avec l'Esprit Consolateur,
A travers les siècles sans fin.}

\vspace{\baselineskip}

\nocturntitle{Premier nocturne}

\gscore{an_veniet_ecce}{\aarub Voici que viendra le Roi, le Très-haut, avec une grande puissance, pour sauver les nations, alléluia.}
\psalmVulgate{1}
\gscore{an_confortate_manus}{\aarub Fortifiez les mains languissantes, prenez courage et dites : Voici, notre Dieu viendra, et il nous sauvera, alléluia.}
\psalmVulgate{2}
\gscore{an_gaudete_omnes}{\aarub Réjouissez-vous tous et livrez-vous à la joie, car voici que le Seigneur de la vengeance viendra, il amènera la rétribution, il viendra lui-même et nous sauvera.}
\psalmVulgate{3}

\versiculus{Ex Sion spécies decóris ejus.}{Deus noster maniféste véniet.}{C’est de Sion que vient l’éclat de sa splendeur.}{Notre Dieu viendra manifestement.}

\patersecreto

\absolutio{Exáudi, Dómine Jesu Christe, preces servórum tuó\sylprep{rum},~\pscross{} et mise\sylprep{rére} \sylac{no}bis:~\psstar{} Qui cum Patre et Spíritu Sancto vivis et regnas in sǽcula sæculórum.}{Exauce, Seigneur Jésus-Christ, les prières de tes serviteurs, et aie pitié de nous, toi qui vis et règnes avec le Père et le Saint-Esprit, dans les siècles des siècles.}

\benedictio{Benedictió\sylprep{ne} \sylprep{per}\sylac{pé}tua~\psstar{} benedícat nos Pater ætérnus.}{Que le Père éternel nous bénisse d'une bénédiction perpétuelle.}

\smalltitle{Première leçon}

\twocoltext{Incipit liber Isaíæ Prophétæ.\\
	\rubric{Is. 1: 1-3}\\
	Vísio Isaíæ fílii Amos, quam vidit super Judam et Jerúsalem, in diébus Ozíæ, Jóatham, Achaz, et Ezechíæ, regum Juda. Audíte, cæli, et áuribus pércipe, terra, quóniam Dóminus locútus est: Fílios enutrívi, et exaltávi: ipsi autem sprevérunt me. Cognóvit bos possessórem suum, et ásinus præsépe dómini sui: Israël autem me non cognóvit, et pópulus meus non intelléxit.
}{
	Commencement du livre du Prophète Isaïe.\\
	Vision d’Isaïe, fils d’Amots,
	--- ce qu’il a vu au sujet de Juda et de Jérusalem,
	au temps d’Ozias, de Yotam, d’Acaz et d’Ézékias, rois de Juda.
Cieux, écoutez; terre, prête l’oreille,
	car le Seigneur a parlé.
J’ai fait grandir des enfants, je les ai élevés, mais ils se sont révoltés contre moi.
Le bœuf connaît son propriétaire, et l’âne, la crèche de son maître.
	Israël ne le connaît pas, mon peuple ne comprend pas.
}

\tuautem

\gscore{re_aspiciens_a_longe}{\rrrub Regardant de loin, voici que je vois venir la puissance de Dieu, et une nuée qui couvre toute la terre. Allez à sa rencontre et dites: Annonce-nous si c’est toi, celui qui doit régner sur le peuple d’Israël. \vvrub Vous tous, fils de la terre, et fils des hommes, ensemble et de concert, riche et pauvre: * Allez à sa rencontre et dites: \vvrub Toi qui gouvernes Israël, regarde, toi qui conduis Joseph comme une brebis: * Annonce-nous si c'est toi. \vvrub Portes, levez vos frontons, élevez-vous, portes éternelles, qu'il entre, le Roi de gloire: * Celui qui doit régner sur le peuple d’Israël.}

\vspace{\baselineskip}

\benedictio{Unigénitus \sylprep{Dei} \sylac{Fí}lius~\psstar{} nos benedícere et adjuváre dignétur.}{Que le Fils unique de Dieu daigne nous bénir et nous secourir.}

\smalltitle{Deuxième leçon}

\twocoltext{\rubric{Is. 1: 4-6}\\
Væ genti peccatríci, pópulo gravi iniquitáte, sémini nequam, fíliis scelerátis: dereliquérunt Dóminum, blasphemavérunt Sanctum Israël, abalienáti sunt retrórsum.
Super quo percútiam vos ultra, addéntes prævaricatiónem? omne caput lánguidum, et omne cor mærens.
A planta pedis usque ad vérticem non est in eo sánitas: vulnus, et livor, et plaga tumens non est circumligáta, nec curáta medicámine, neque fota óleo.
}{
Malheur à vous, nation pécheresse, peuple chargé de fautes,
	engeance de malfaiteurs, fils pervertis!
Ils abandonnent le Seigneur,
	ils méprisent le Saint d’Israël, ils lui tournent le dos.
Où donc faut-il vous frapper encore, vous qui multipliez les reniements?
	Toute la tête est malade, tout le cœur est atteint;
	de la plante des pieds à la tête, plus rien n’est intact:
	partout blessures, contusions, plaies ouvertes,
	qui ne sont ni pansées, ni bandées, ni soignées avec de l’huile.
}

\gscore{re_aspiciebam}{\rrrub Je regardais dans la vision de nuit et voici comme le Fils d’un homme qui venait dans les nuées du Ciel ; et il lui fut donné le royaume et l’honneur : et tous les peuples, tribus et langues le serviront. \vvrub Sa puissance est une puissance éternelle, qui ne lui sera pas ôtée, et son royaume ne sera pas détruit.}

\vspace{\baselineskip}

\benedictio{Spíritus \sylprep{Sancti} \sylac{grá}tia~\psstar{} illúminet sensus et corda nostra.}{Que la grâce du Saint-Esprit illumine nos esprits et nos cœurs.}

\smalltitle{Troisième leçon}

\twocoltext{\rubric{Is. 1:7-9}\\
Terra vestra desérta, civitátes vestræ succénsæ igni: regiónem vestram coram vobis aliéni dévorant, et desolábitur sicut in vastitáte hostíli.
Et derelinquétur fília Sion ut umbráculum in vínea, et sicut tugúrium in cucumerário, et sicut cívitas quæ vastátur.
Nisi Dóminus exercítuum reliquísset nobis semen, quasi Sódoma fuissémus, et quasi Gomórrha símiles essémus.
}{
Votre pays n’est que désolation, vos villes sont consumées par le feu;
	votre terre, des étrangers la dévorent sous vos yeux,
	c’est une désolation, comme un désastre venu des étrangers.
Ce qui reste de la fille de Sion est comme une hutte dans une vigne,
	comme un abri dans un potager, comme une ville assiégée.
Si le Seigneur de l’univers ne nous avait laissé un petit reste,
	nous serions comme Sodome, nous ressemblerions à Gomorrhe.
}

\gscore{re_missus_est_gabriel}{\rrrub L’Ange Gabriel fut envoyé à Marie, vierge qu’avait épousée Joseph, lui annonçant la parole : mais la Vierge s’effraya de la lumière. Ne craignez point, Marie, vous avez trouvé grâce devant Dieu : Voilà que vous concevrez et enfanterez, et il sera appelé le Fils du Très-Haut.
\vvrub Le Seigneur lui donnera le trône de David son père et il régnera éternellement sur la maison de Jacob.}

\vspace{\baselineskip}
\vfill

\nocturntitle{Deuxième nocturne}

\gscore{an_gaude_et_laetare}{\aarub Réjouis-toi, réjouis-toi, fille de Jérusalem: voici que ton Roi vient à toi: Sion, ne crains pas, car ton salut viendra bientôt.}

\psalmVulgate{8}
\gscore{an_rex_noster_adveniet}{\aarub Notre Roi, le Christ, viendra, lui que Jean a prédit être l’Agneau qui doit venir.}
\psalmVulgate{9-i}
\gscore{an_ecce_venio}{\aarub Voici que je viens bientôt, et ma récompense est avec moi, dit le Seigneur ; c’est de donner à chacun selon ses œuvres.}
\psalmVulgate{9-ii}

\vspace{\baselineskip}

\versiculus{Emítte Agnum, Dómine, Dominatórem terræ.}{De Petra desérti ad montem fíliæ Sion.}{Envoyez, Seigneur, l’Agneau dominateur de la terre.}{De la pierre du désert à la montagne de la fille de Sion.}

\vspace{\baselineskip}

\patersecreto

\absolutio{Ipsíus píetas et misericórdi\sylprep{a} \sylprep{nos} \sylac{ád}juvet,~\psstar{} qui cum Patre et Spíritu Sancto vivit et regnat in sǽcula sæculórum.}{Qu'il nous secoure par sa bonté et sa miséricorde, celui qui, avec le Père et le Saint-Esprit, vit et règne dans les siècles des siècles.}

\benedictio{Deus Pa\sylprep{ter} \sylprep{om}\sylac{ní}potens~\psstar{} sit nobis propítius et clemens.}{Que Dieu le Père tout-puissant soit pour nous propice et plein de clémence.}

\smalltitle{Quatrième leçon}

\twocoltext{Sermo sancti Leónis Papæ.\\
\rubric{Sermo 8 de jejunio decimi mensis et eleemosynis}\\
Cum de advéntu regni Dei, et de mundi fine ac témporum, discípulos suos Salvátor instrúeret, totámque Ecclésiam suam in Apóstolis erudíret: Cavéte, inquit, ne forte gravéntur corda vestra in crápula, et ebrietáte, et cogitatiónibus sæculáribus. Quod útique præcéptum, dilectíssimi, ad nos speciálius pertinére cognóscimus, quibus denuntiátus dies, etiámsi est occúltus, non dubitátur esse vicínus.
}{Sermon de saint Léon, Pape.\\
\rubric{Semon n°8, sur le jeûne du dixième mois et les aumônes}\\
Le Sauveur, instruisant ses disciples au sujet de l’avènement du royaume de Dieu, ainsi que de la fin du monde et des temps, et, en la personne de ses Apôtres, instruisant toute son Église, leur dit : «Faites attention, de peur que vos cœurs ne s’appesantissent dans l’excès du manger et du boire et les soins de cette vie.» Nous savons, très chers, que ce précepte nous regarde tout spécialement, puisque l’on ne doute guère que ce jour annoncé, quoique encore caché, ne soit bien proche.}

\gscore{re_ave_maria}{\rrrub Je te salue, Marie, pleine de grâce, le Seigneur est avec toi: l’Esprit Saint viendra sur toi, et la puissance du Très-Haut te couvrira de son ombre ; c’est pourquoi celui qui naîtra de toi sera appelée le Fils de Dieu. \vvrub Comment cela se fera-t-il, car je ne connais pas d’homme ? Et l’Ange, répondant, lui dit : l'Esprit Saint viendra sur toi.}

\benedictio{Chris\sylprep{tus} \sylprep{per}\sylac{pé}tuæ~\psstar{} det nobis gáudia vitæ.}{Que le Christ nous donne les joies de l'éternelle vie.}

\smalltitle{Cinquième leçon}

\twocoltext{Ad cujus advéntum omnem hóminem cónvenit præparári: ne quem aut ventri déditum, aut curis sæculáribus invéniat implicátum. Quotidiáno enim, dilectíssimi, experiménto probátur, potus satietáte áciem mentis obtúndi, et cibórum nimietáte vigórem cordis hebetári; ita ut delectátio edéndi étiam córporum contrária sit salúti, nisi rátio temperántiæ obsístat illécebræ, et quod futúrum est óneri, súbtrahat voluptáti.
}{
Il convient que tout homme se prépare à l’avènement du Sauveur ; de crainte qu’il ne le trouve livré à la gourmandise, ou embarrassé dans les soucis du monde. Il est prouvé, par une expérience de tous les jours, que la vivacité de l’esprit s’altère par l’excès du boire, et que l’énergie du cœur est affaiblie par une trop grande quantité d’aliments. Le plaisir de manger peut devenir nuisible, même à la santé du corps, si la raison et la tempérance ne le modèrent, ne résistent à l’attrait, et ne retranchent au plaisir ce qui serait superflu.
}

\gscore{re_salvatorem_exspectamus}{\rrrub Nous attendons le Sauveur, notre Seigneur Jésus-Christ, qui réformera le corps de notre humilité, en le conformant à son corps glorieux. \vvrub Vivons sobrement, justement et pieusement en ce monde, attendant la bienheureuse espérance, et l’avènement de la gloire du grand Dieu.}

\benedictio{Ignem su\sylprep{i} \sylprep{a}\sylac{mó}ris~\psstar{} accéndat Deus in córdibus nostris.}{Que Dieu allume dans nos cœurs le feu de son amour.}

\smalltitle{Sixième leçon}

\twocoltext{
Quamvis enim sine ánima nihil caro desíderet, et inde accípiat sensus, unde sumit et motus: ejúsdem tamen est ánimæ, quædam sibi súbditæ negáre substántiæ, et interióri judício ab inconveniéntibus exterióra frenáre: ut a corpóreis cupiditátibus sǽpius líbera, in aula mentis possit divínæ vacáre sapiéntiæ: ubi omni strépitu terrenárum silénte curárum, in meditatiónibus sanctis, et in delíciis lætétur ætérnis.
}{
Car, bien que, sans l’âme, la chair ne désirerait rien, et que c’est d’elle qu’elle reçoit la sensibilité, comme elle en reçoit le mouvement, il est cependant du devoir de cette âme de refuser certaines choses à la substance matérielle qui lui est assujettie. Par un jugement intérieur, elle doit tenir ses sens extérieurs éloignés de ce qui ne lui convient pas, afin qu’étant presque constamment détachée des désirs corporels, elle puisse vaquer à l’étude de la sagesse divine dans le palais de l’intelligence, où le bruit des sollicitudes terrestres ne se faisant plus entendre, elle se réjouit dans des méditations saintes, à la pensée des délices éternels.
}

\gscore{re_obsecro_domine}{\rrrub Je t'en conjure, Seigneur, envoie celui qui doit venir: vois l’affliction de ton peuple: comme tu l’as promis, viens, et délivre-nous. \vvrub Toi qui gouvernes Israël, viens, toi qui conduis Joseph comme une brebis, toi qui es assis au-dessus des Chérubins.}

\nocturntitle{Troisième nocturne}

\gscore{an_gabriel_angelus_ave}{\aarub L’Ange Gabriel parla à Marie, disant : Je te salue, pleine de grâce, le Seigneur est avec toi, tu es bénie entre les femmes.}
\psalmVulgate{9-iii}
\gscore{an_maria_dixit}{\aarub Marie dit : que veut dire cette salutation ? Car mon âme a été troublée, et que je dois enfanter un Roi qui ne violera pas ma virginité.}
\psalmVulgate{9-iv}

\vfill

\gscore{an_in_adventu}{\aarub En l’avènement du souverain Roi, que les cœurs des hommes soient purifiés, que nous marchions dignement à sa rencontre : car voici qu’il vient, et il ne tardera pas.}

\psalmVulgate{10}

\versiculus{Egrediétur Dóminus de loco sancto suo.}{Véniet ut salvet pópulum suum.}{Le Seigneur sortira de son lieu saint.}{Il viendra pour sauver son peuple.}

\patersecreto

\absolutio{A vínculis peccató\sylprep{rum} \sylprep{nos}\sylac{tró}rum~\psstar{} absólvat nos omnípotens et miséricors Dóminus.}{Que le Dieu tout-puissant et miséricordieux daigne nous délivrer des liens de nos péchés.}

\benedictio{Evangé\sylprep{lica} \sylac{lé}ctio~\psstar{} sit nobis salus et protéctio.}{Que la lecture du saint Évangile nous soit salut et protection.}

\smalltitle{Septième leçon}

\twocoltext{Léctio sancti Evangélii secúndum Lucam.\\
\rubric{Lc. 21: 25-33}\\
In illo témpore: Dixit Jesus discípulis suis: Erunt signa in sole, et luna, et stellis, et in terris pressúra géntium. Et réliqua.\\
~\\
Homilía sancti Gregórii Papæ.\\
\rubric{Homilia 1 in Evangelia}\\
Dóminus ac Redémptor noster parátos nos inveníre desíderans, senescéntem mundum quæ mala sequántur denúntiat, ut nos ab ejus amóre compéscat. Appropinquántem ejus términum quantæ percussiónes prævéniant, innotéscit: ut, si Deum metúere in tranquillitáte nólumus, saltem vicínum ejus judícium vel percussiónibus attríti timeámus.
}{
Lecture du saint Évangile selon saint Luc.\\
En ce temps-là : Jésus dit à ses disciples : il y aura des signes dans le soleil, dans la lune et dans les astres, et, sur la terre, une angoisse des nations. Et le reste.\\
~\\
Homélie de saint Grégoire, Pape.\\
\rubric{Homélie n°1 sur les Évangiles}\\
Notre Seigneur et Rédempteur, désirant nous trouver prêts, nous annonce les maux qui doivent accompagner la vieillesse du monde, pour nous détourner de son amour. Il nous fait connaître les maux qui précéderont sa fin prochaine, afin que, si nous ne voulons pas craindre Dieu dans la tranquillité, nous redoutions au moins son prochain jugement et soyons comme atterrés par les coups de sa justice.
}

\gscore{re_ecce_virgo}{\rrrub Voici, dit le Seigneur, que la Vierge concevra et enfantera un fils: et son nom sera appelé Admirable, Dieu, Fort. \vvrub Il s’assiéra sur le trône de David, et sur son royaume pour l’éternité.}

\benedictio{Diví\sylprep{num} \sylprep{au}\sylac{xí}lium~\psstar{} máneat semper nobíscum.}{Que le secours divin demeure toujours avec nous.}

\smalltitle{Huitième leçon}

\twocoltext{
Huic étenim lectióni sancti Evangélii, quam modo vestra fratérnitas audívit, paulo supérius Dóminus præmísit, dicens: Exsúrget gens contra gentem, et regnum advérsus regnum: et erunt terræmótus magni per loca, et pestiléntiæ, et fames. Et quibúsdam interpósitis, hoc, quod modo audístis, adjúnxit: Erunt signa in sole, et luna, et stellis, et in terris pressúra géntium præ confusióne sónitus maris, et flúctuum. Ex quibus profécto ómnibus ália jam facta cérnimus, ália in próximo ventúra formidámus.
}{
Un peu avant le passage du saint Évangile que votre fraternité a entendu tout à l’heure, le Seigneur a dit d’abord : « Une nation se soulèvera contre une nation, un royaume contre un royaume. Il y aura de grands tremblements de terre en divers lieux, et des pestes et des famines. » Et, un peu plus loin, il ajoute ce que vous venez également d’entendre : « II y aura des signes dans le soleil, dans la lune et dans les étoiles, et, sur la terre, la détresse des nations, à cause du bruit confus de la mer et des flots. » De toutes ces choses, les unes, nous les voyons déjà accomplies, les autres, nous craignons de les voir arriver bientôt.
}

\gscore{re_audite_verbum}{\rrrub Écoutez, Nations, la parole du Seigneur, et annoncez-la aux extrémités de la terre :
et aux îles qui sont au loin, dites : Notre Sauveur viendra. \vvrub Annoncez et faites entendre, parlez et criez.}

\benedictio{Ad societátem cívium \sylprep{super}\sylac{nó}rum~\psstar{} perdúcat nos Rex Angelórum.}{Que le Roi des Anges nous fasse parvenir à la société des citoyens célestes.}

\smalltitle{Neuvième leçon}

\twocoltext{
Nam gentem contra gentem exsúrgere, earúmque pressúram terris insístere, plus jam in nostris tempóribus cérnimus, quam in codícibus légimus. Quod terræmótus urbes innúmeras óbruat, ex áliis mundi pártibus scitis quam frequénter audívimus. Pestiléntias sine cessatióne pátimur. Signa vero in sole, et luna, et stellis, adhuc apérte mínime vídimus: sed quia et hæc non longe sint, ex ipsa jam áëris immutatióne collígimus.
}{
Que les nations se soulèvent les unes contre les autres, que la consternation soit parmi les peuples, nous le voyons à notre époque, plus que jamais on ne le vit autrefois. Que des tremblements de terre renversent des villes innombrables en d’autres parties du monde, vous savez combien de fois nous l’avons entendu dire. La peste ne cesse de nous affliger. Quant aux signes dans le soleil, la lune et les étoiles, jusqu’ici nous n’en voyons pas ; mais, le changement que nous remarquons dans l’atmosphère, nous permet de présumer qu’ils ne tarderont pas à se manifester.
}

\gscore{re_ecce_dies}{\rrrub Voilà que des jours viennent, dit le Seigneur, et je susciterai à David un germe juste ; un Roi régnera, il sera sage, et il rendra le jugement et la justice sur la terre : Et voici le nom dont on l’appellera: Le Seigneur, notre juste. \vvrub En ces jours- là Juda sera sauvé, et Israël habitera en assurance.}

\dominusvobiscum

\oratio{Orémus. Excita, quǽsumus, Dómine, poténtiam tuam, et veni:~\pscross{} ut ab imminéntibus peccatórum nostrórum perículis, te mereámur protegénte éripi,~\psstar{} te liberánte salvári:
Qui vivis et regnas cum Deo Patre, in unitáte Spíritus Sancti, Deus, per ómnia sǽcula sæculórum.}{Réveille ta puissance, Seigneur, et viens: dans le grand péril où nous sommes à cause de nos péchés, sois le défenseur qui nous délivre et le libérateur qui nous sauve.
Toi qui vis et règnes avec le Père dans l'unité du Saint-Esprit, Dieu, pour les siècles des siècles.}

\dominusvobiscum

\smallscore{or_benedicamus_adventus}{\vvrub Bénissons le Seigneur. \rrrub Nous rendons grâces à Dieu.}

\fideliumanimae

\end{document}
