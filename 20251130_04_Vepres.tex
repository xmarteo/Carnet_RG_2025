% !TEX TS-program = lualatex
% !TEX encoding = UTF-8

\documentclass[Carnet_RG_2025.tex]{subfiles}

\ifcsname preamble@file\endcsname
  \setcounter{page}{\getpagerefnumber{M-20251130_04_Vepres}}
\fi

\begin{document}
\bigtitle{Deuxièmes vêpres\\du premier dimanche de l'Avent}{Dimanche 30 novembre}{Vêpres}

\smallscore{or_dia_FO_festivus}{}

\gscore{hy_conditor_alme_siderum}{}
\hymntranslation{\colored{B}ienfaisant Créateur des Cieux,
lumière éternelle des croyants,
Rédempteur de tous les hommes,
ô Jésus, écoutez les vœux de ceux qui vous prient.\\
\colored{A}fin d’empêcher la terre
de périr par les pièges du démon, dans l’élan
de votre amour, vous vous êtes fait
le remède des maux de ce monde coupable.\\
\colored{P}our expier, sur la croix,
le crime commun des hommes,
ô victime innocente,
vous sortez de l’auguste sein de la Vierge.\\
\colored{À} la vue de votre gloire et de votre puissance,
et dès que votre nom se fait entendre,
au Ciel et dans les enfers
tout fléchit le genou avec crainte.\\
\colored{J}uge souverain du dernier jour,
nous vous en supplions,
daignez nous défendre de nos ennemis,
par les armes de la grâce céleste.\\
\colored{P}uissance, honneur, louange et gloire
à Dieu le Père et à son Fils,
ainsi qu’au saint Consolateur
dans les siècles des siècles.}

\gresetnabc{1}{invisible} %% début d'une section où on cache du NABC synthetique

\gscore{an_iucundare_filia}{\aarub Exulte de toutes tes forces, fille de Sion ! Pousse des cris de joie, fille de Jérusalem !}
\psalmNova{109}

\gscore{an_rex_noster_adveniet}{\aarub Vient le Messie, notre Roi, celui que Jean-Baptiste annonçait : Voici l’Agneau de Dieu !}
\psalmNova{113A}

\gscore{an_ecce_venio}{\aarub Voici que je viens sans tarder, et j’apporte avec moi le salaire que je vais donner à chacun selon ce qu’il a fait.}
\canticumNova{Ap19}{de l'Apocalypse}

\gresetnabc{1}{visible} %% fin d'une section où on cache du NABC synthetique


\lectiobrevis{Phil. 4: 4-5}{Gaudéte in Dómino semper. Íterum dico: Gaudéte!
Modéstia vestra nota sit ómnibus homínibus. Dóminus
prope.}{ Soyez toujours dans la joie du Seigneur ; laissez-moi vous le redire : soyez dans la joie. Que votre sérénité soit connue de tous les hommes. Le Seigneur est proche. }

\gscore{rb_ostende_nobis}{\rr Fais-nous voir, Seigneur, * ton amour. Fais-nous. \\\vv Et donne-nous ton salut. Ton amour. Gloire au Père. Fais-nous.}

\gscore{an_spiritus_sanctus}{\rubric{Lc. 1: 30, 35} \aarub L'Esprit Saint viendra sur toi, Marie: ne crains pas, tu portera le Fils de Dieu, alléluia.}
\canticumNova{Magnificat}{de Marie}

\smallscore{ORPreces}{}
\twocoltext{Redemptórem nostrum Iesum Christum, qui est via, véritas et vita, supplíciter rogémus, dicéntes:
Veni et mane nobíscum, Dómine.
Iesu, Fili Altíssimi, qui Vírgini Maríæ per Gabriélem annuntiátus es,
–– veni ad regnándum super plebem tuam in ætérnum. Kyrie.
Tu, Sancte Dei, cui Præcúrsor in sinu Elísabeth exsultávit,
–– veni ad dandum univérso mundo gáudium salútis. Kyrie
Iesu Salvátor, cuius nomen Ioseph viro iusto ab Ángelo revelátum est,
–– veni ad pópulum tuum salvum faciéndum a peccátis eórum. Kyrie.
Lumen mundi, quod Símeon et omnes iusti exspectábant,
–– veni ad consolándum nos. Kyrie.
Óriens indefíciens, quem nos ex alto visitatúrum prædíxit Zacharías,
–– veni ad illuminándum illos, qui in umbra mortis sedent. Kyrie.}{Supplions Jésus Christ, notre Rédempteur, lui qui est la voie, la vérité et la vie :
Kyrie eleison.
Jésus, Fils du Très-Haut, toi qui as été annoncé à la Vierge Marie par l’ange Gabriel,
— viens régner sur ton peuple pour toujours. Kyrie.
Toi, le Saint de Dieu, pour qui le Précurseur a tressailli d’allégresse dans le sein d’Élisabeth,
— viens donner la joie du salut à tout l’univers. Kyrie.
Jésus Sauveur, toi dont le Nom a été révélé par l’ange à Joseph, l’homme juste,
— viens sauver ton peuple de ses péchés. Kyrie.
Lumière du monde, toi qu’attendaient Syméon et tous les justes,
— viens nous consoler. Kyrie.
Astre d’en haut, lumière sans déclin, Zacharie a prédit ta venue :
— viens illuminer ceux qui habitent l’ombre de la mort. Kyrie.}

\smallscore{or_pater_festivus}{} % à adapter si nécessaire

\oratio{Da, quæsumus, omnípotens Deus, hanc tuis fidélibus voluntátem, ut, Christo tuo veniénti iustis opéribus occurréntes, eius déxteræ sociáti, regnum mereántur possidére cæléste. Per Dóminum.}{Donne à tes fidèles, Dieu tout-puissant, la volonté d’aller par les chemins de la justice à la rencontre de celui qui vient, le Christ, afin qu’ils soient admis à sa droite et méritent d’entrer en possession du Royaume des cieux. Par Jésus Christ.}

\blessing

\smallscore{or_benedicamus_adventus}{\vvrub Bénissons le Seigneur. \rrrub Nous rendons grâces à Dieu.}

\end{document}
