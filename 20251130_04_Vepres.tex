% !TEX TS-program = lualatex
% !TEX encoding = UTF-8

\documentclass[Carnet_RG_2025.tex]{subfiles}

\ifcsname preamble@file\endcsname
  \setcounter{page}{\getpagerefnumber{M-20251130_04_Vepres}}
\fi

\begin{document}
\bigtitle{Deuxièmes vêpres\\du premier dimanche de l'Avent}{Dimanche 30 novembre}{Vêpres}

\smallscore{or_dia_FO_festivus}{}
\vspace{-\baselineskip}
\smalltitle{Conduit}
Montpellier, Bibliothèque de la Faculté de Médecine H 196.\\
Manuscrit parisien de l’Ars antiqua, XIIIe siècle.

\columnratio{0.4}
\twocoltext{\colored{D}eus, in adiutórium\\
inténde laborántium;\\
ad dolóris remédium,\\
festína in auxílium.\\
\colored{U}t chorus noster psállere\\
possit et laudes dícere\\
tibi, Christe, rex glórie:\\
glória tibi, Dómine!\\
\colored{I}n te, Christe, credéntium\\
misereáris ómnium,\\
qui es Deus in sécula\\
seculórum in glória.\\
\colored{A}men, amen, alleluya.\\
Amen, amen, alleluya.\\
Amen, amen, alleluya.\\
Amen, amen, alleluya.}{\vspace{-0.1mm}\colored{D}ieu, viens au secours\\
de ceux qui peinent ;\\
afin de porter remède à leur douleur,\\
hâte-toi de les secourir.\\
\colored{Q}ue notre chœur puisse te chanter des psaumes\\
et te proclamer des louanges,\\
ô Christ, roi de gloire :\\
gloire à toi, Seigneur !\\
\colored{A}ie pitié, ô Christ,\\
de tous ceux qui croient en toi,\\
toi qui es Dieu pour les siècles des siècles\\
dans la gloire.\\
\colored{A}men, amen, alléluia. / Amen, amen, alléluia. / Amen, amen, alléluia. / Amen, amen, alléluia}
\columnratio{0.5}
\pagebreak

\gscore{hy_conditor_alme_siderum}{}
\hymntranslation{\colored{B}ienfaisant créateur des cieux,\\
Pour toujours soleil des croyants,\\
Ô rédempteur du genre humain,\\
Christ, entends nos voix suppliantes.\\
\colored{C}ompatissant devant la mort\\
Dont notre siècle périssait,\\
Tu sauvas le monde épuisé,\\
Donnant aux pécheurs le remède.\\
\colored{C}omme un époux dans le soleil\\
Quand le monde allait vers sa nuit,\\
Tu sortis du jardin fermé\\
D'une Vierge, mère et bénie.\\
\colored{D}evant ta souveraineté\\
Tout être fléchit le genou ;\\
Tout dans le ciel et ici-bas\\
S'avoue soumis à ta puissance.\\
\colored{N}otre foi t'implore, ô très saint,\\
Toi qui viendras juger ce temps,\\
De nous protéger aujourd'hui\\
Du trait de l'ennemi perfide.\\
\colored{Ô} Christ, ô Roi plein de bonté,\\
Gloire à toi et gloire à ton Père,\\
Avec l'Esprit Consolateur,\\
A travers l'infini des siècles !}

\gresetnabc{1}{invisible} %% début d'une section où on cache du NABC synthetique

\gscore{an_iucundare_filia}{\aarub Exulte de toutes tes forces, fille de Sion ! Pousse des cris de joie, fille de Jérusalem !}
\psalmNova{109}

\gscore{an_rex_noster_adveniet}{\aarub Vient le Messie, notre Roi, celui que Jean-Baptiste annonçait : Voici l’Agneau de Dieu !}
\psalmNova{113A}

\vfill

\gscore{an_ecce_venio}{\aarub Voici que je viens sans tarder, et j’apporte avec moi le salaire que je vais donner à chacun selon ce qu’il a fait.}
\pagebreak
\canticumNova{Ap19}{de l'Apocalypse}

\gresetnabc{1}{visible} %% fin d'une section où on cache du NABC synthetique

\vspace{3\baselineskip}

\lectiobrevis{Phil. 4: 4-5}{Gaudéte in Dómino semper. Íterum dico: Gaudéte!
Modéstia vestra nota sit ómnibus homínibus. Dóminus
prope.}{ Soyez toujours dans la joie du Seigneur ; laissez-moi vous le redire : soyez dans la joie. Que votre sérénité soit connue de tous les hommes. Le Seigneur est proche. }

\newpage

\gscore{rb_ostende_nobis}{\rr Fais-nous voir, Seigneur, * ton amour. Fais-nous. \\\vv Et donne-nous ton salut. Ton amour. Gloire au Père. Fais-nous.}

\gscore{an_spiritus_sanctus}{\rubric{Lc. 1: 30, 35} \aarub L'Esprit Saint viendra sur toi, Marie: ne crains pas, tu portera le Fils de Dieu, alléluia.}
\canticumNova{Magnificat}{de Marie}

\smallscore{an_spiritus_sanctus}{}

\smallscore{ORPreces}{}
\twocoltext{Redemptórem nostrum Iesum Christum, qui est via, véritas et vi\sylprep{ta},~\pscross{} supplíciter rogé\sylprep{mus}, \sylprep{di}\sylac{cén}tes: Kyrie eléison.\\
Iesu, Fili Altíssimi, qui Vírgini Maríæ per Gabriélem annuntiá\sylprep{tus} \sylprep{es},~\pscross{} veni ad regnándum super plebem tuam \sylprep{in} \sylprep{æ}\sylac{tér}num. Kyrie.\\
Tu, Sancte Dei, cui Præcúrsor in sinu Elísabeth exsultá\sylprep{vit},~\pscross{} veni ad dandum univérso mundo gáudi\sylprep{um} \sylprep{sa}\sylac{lú}tis. Kyrie.\\
Iesu Salvátor, cuius nomen Ioseph viro iusto ab Ángelo revelá\sylprep{tum} \sylprep{est},~\pscross{} veni ad pópulum tuum salvum faciéndum a peccá\sylprep{tis} \sylprep{e}\sylac{ó}rum. Kyrie.\\
Lumen mundi, quod Símeon et omnes iusti exspectá\sylprep{bant},~\pscross{} veni ad \sylprep{conso}\sylac{lán}dum nos. Kyrie.\\
Óriens indefíciens, quem nos ex alto visitatúrum prædíxit Zacharí\sylprep{as},~\pscross{} veni ad illuminándum illos, qui in umbra \sylprep{mortis} \sylac{se}dent. Kyrie.}{Supplions Jésus Christ, notre Rédempteur, lui qui est la voie, la vérité et la vie : Kyrie eleison.\\
Jésus, Fils du Très-Haut, toi qui as été annoncé à la Vierge Marie par l’ange Gabriel,
— viens régner sur ton peuple pour toujours. Kyrie.\\
Toi, le Saint de Dieu, pour qui le Précurseur a tressailli d’allégresse dans le sein d’Élisabeth,
— viens donner la joie du salut à tout l’univers. Kyrie.\\
Jésus Sauveur, toi dont le Nom a été révélé par l’ange à Joseph, l’homme juste,
— viens sauver ton peuple de ses péchés. Kyrie.\\
Lumière du monde, toi qu’attendaient Syméon et tous les justes,
— viens nous consoler. Kyrie.\\
Astre d’en haut, lumière sans déclin, Zacharie a prédit ta venue :
— viens illuminer ceux qui habitent l’ombre de la mort. Kyrie.}

\smallscore{or_pater_festivus}{} % à adapter si nécessaire

\oratio{Da, quæsumus, omnípotens Deus, hanc tuis fidélibus voluntátem, ut, Christo tuo veniénti iustis opéribus occurréntes, eius déxteræ sociáti, regnum mereántur possidére cæléste. Per Dóminum.}{Donne à tes fidèles, Dieu tout-puissant, la volonté d’aller par les chemins de la justice à la rencontre de celui qui vient, le Christ, afin qu’ils soient admis à sa droite et méritent d’entrer en possession du Royaume des cieux. Par Jésus Christ.}

\blessing

\smallscore{benedicamus_domino_sylvain_dieudonne}{\vvrub Bénissons le Seigneur. \rrrub Nous rendons grâces à Dieu.}

\end{document}
